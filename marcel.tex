\section{Hilbert-Eindeutigkeits-Methode}

Nachdem wir also gesehen haben, wann ein lineares 
Kontrollsystem  kontrollierbar ist und wie dann eine, unter
einem gewissen Aspekt, optimale Steuerung aussieht, wollen wir uns
einen etwas allgemeineren Rahmen der Problemstellung widmen. Dazu
sei daran erinnert, wie in Beweis \ref{Kontrollierbarkeit} die Steuerung
$\bar{u}$ definiert wurde:
$$
\bar{u}(\tau):=B(\tau)^TR(T_{1},\tau)^T\mathfrak{C}^{-1}(x_{1}-R(T_{1},T_{0})x_{0}),~~\forall\tau\in[T_{0},T_{1}]
$$


Bei der Berechnung dieser Steuerung treten einige Probleme auf, die
wir im Folgenden betrachten wollen.

Sei also weiter das Kontrollsystem
$$
\dot{x}(t)=A(t)x(t)+B(t)u(t),~~~~x(T_{0})=0
$$
zu Grunde gelegt.

\subsection*{Die Erreichbarkeitsmenge}

Die Steuerung $\bar{u}$ konnte nur definiert werden, da die Gramsche
Kontrollmatrix $\mathfrak{C}$ invertierbar war, was genau dann der
Fall ist, wenn das Kontrollsystem  kontrollierbar ist.
Dies ist natürlich im Allgemeinen nicht immer der Fall, wie das folgende
einfache Beispiel zeigt:
\begin{Beispiel}
[nicht kontrollierbares System]\label{bsp1}

Sei ein zeitunabhängiges lineares Kontrollsystem gegeben durch die
Matrizen
$$
A=\begin{pmatrix}1 & 0\\
0 & 1
\end{pmatrix}\text{\,\,\,\ und\,\,}\, B=\begin{pmatrix}1\\
1
\end{pmatrix}
$$
Dann folgt direkt mit Hilfe des Kalman-Rang-Kriteriums, dass das System
nicht kontrollierbar ist, da $A^{i}=A,\,\forall i\in\mathbb{N}$
und damit $rang(B\,|\, AB)=1$.
\end{Beispiel}
Darüber hinaus ist selbst bei kontrollierbaren Systemen
die Gramsche Kontrollmatrix nicht ohne weiteres berechenbar und/oder
der Aufwand für Bestimmung und Invertierung in keinem akzeptablem
Rahmen (Rechenaufwand Matrixinvertierung $\mathcal{O}(n^{3})$). 

Um dennoch weitere Aussagen über solche Systeme treffen zu können,
die eben nicht kontrollierbar sind, führen wir einen neuen
Begriff ein:


\begin{Definition}[Erreichbarkeitsmenge]
Wir definieren die Menge 
\begin{align*}
\mathcal{R}(t):=&\{x_{1}\in\mathbb{R}^{n}|\text{ es ex. eine Steuerung }u\\
&x\text{ Lösung des Kontrollsystems und es gilt }x(t)=x_{1}\}
\end{align*}
\end{Definition}

Das heißt $\text{\ensuremath{\mathcal{R}}(t) ist die Menge der Zustände (oder Punkte), die von \ensuremath{x_{0}=0}}$
in der Zeit $t=T_{1}-T_{0}$ durch alle möglichen Steuerungen $u$
erreicht werden können.

Für unser Beispiel (\ref{bsp1}) lässt sich die Erreichbarkeitsmenge
wie folgt bestimmen:

Die Lösung des Anfangswertproblems hat die Form
$$
x(t)=e^{A(t-T_{0})}x_{0}+\int_{T_{0}}^{t}e^{A(t-s)}Bu(s)ds=\int_{T_{0}}^{t}e^{A(t-s)}\begin{pmatrix}1\\
1
\end{pmatrix}u(s)ds
$$
Und damit, wegen der Diagonalität von $A(t)$, ergibt sich für die
Lösungskomponenten
$$
\begin{pmatrix}x_{1}(t)\\
x_{2}(t)
\end{pmatrix}=\begin{pmatrix}\int_{T_{0}}^{t}e^{(t-s)}u(s)ds\\
\int_{T_{0}}^{t}e^{(t-s)}u(s)ds
\end{pmatrix}
$$
Das heißt, beide Lösungskomponente sind für jede Steuerung $u$ zu
jeder Zeit $t>T_{0}$ identisch:

%\vspace{5cm}
\begin{center}
\begin{figure}[ht]
\includegraphics[clip,scale=0.7]{R}\caption{Erreichbarkeitsmenge aus Beispiel (\ref{bsp1})}
\end{figure}
\end{center}
%\vspace{5cm}

Die Erreichbarkeitsmenge eines Kontrollsystems hat einige nützliche
Eigenschaften, die hier zwar kurz erwähnt werden sollen, jedoch nicht
bewiesen werden, da wir sie in unserem Kontext nicht verwenden.

\begin{Lemma}
Eigenschaften der Erreichbarkeitsmenge

\begin{itemize}
\item $\mathcal{R}(t)$ ist für alle $t\geq0$ ein Untervektorraum des $\mathbb{R}^{n}$.
\item $\mathcal{R}(t)=\mathcal{R}(s)$ für alle $s,t>0.$
\end{itemize}
\end{Lemma}
Das heißt also, die Erreichbarkeitsmenge $x_{0}+\mathcal{R}(t)$ ist
ein affiner Unterraum mit der Dimension von $\mathcal{R}(t)$ und
$\mathcal{R}(t)$ ist zeitunabhängig - man kann einen Zustand, falls
er erreichbar ist, in jeder Zeit erreichen. Wir schreiben im weiteren
Verlauf kurz $\mathcal{R}$ statt $\mathcal{R}(t)$.

\subsection*{Hilbert-Eindeutigkeits-Methode (HUM)}

Ziel ist es jetzt einen Weg zu finden, um eine Steuerungsfunktion
zu bestimmen, die unabhängig von der Kontrollierbarkeit, leicht zu
berechnen und in einem noch zu definierenden Sinne \emph{optimal}
ist. Einen solchen Weg bzw. Algorithmus liefert uns die \emph{Hilbert-Eindeutigkeits-Methode(HUM)}.

Dazu führen wir zunächst ein Differentialgleichungssystem ein:

\begin{Definition}[adjungiertes System]


Wir definieren für ein $ \phi_{1}\in\mathbb{R}^{n}$ das adjungierte
zeitumgekehrte homogene Kontrollsystem
$$
\dot{\phi}=-A(t)^T\phi,~~~~\phi(T_{1})=\phi_{1}
$$
und bezeichnen die Lösung mit $\phi(t)$.\end{Definition}

Die Bezeichnung \emph{zeitumgekehrt} wird durch die Tatsache gerechtfertigt, dass
die Lösung dieses Anfangswertproblems die Lösungsbahn des Anfangswertproblems  $\dot{\phi}=A(t)\phi,~~\phi(T_{1})=\phi_{1}$ genau umgekehrt durchläuft.
Um nun die HUM formulieren zu können, benötigen wir die folgende, in
einem Korollar zusammengefasste, Identität:

\begin{Korollar}
Sei $x : [t_0,t_1] \rightarrow \mathbb R^n$ die Lösung des Anfangswertproblems
\begin{align*} \label{hum1} \tag{$\star$}
\dot x = A(t)x+B(t)u(t) ~~~~ \forall x(t_0)=0
\end{align*}
Sei außerdem $\phi_1 \in \mathbb R^n$ und $\phi : [t_0,t_1] \rightarrow \mathbb R^n $ die Lösung des  adjungierten Systems
\begin{align*}
\dot \phi = -A(t)^T\phi ~~~~ \phi(t_1)=\phi_1
\end{align*}
Dann gilt:
\begin{align*}
x(t_1) \cdot \phi_1 = \int_{t_0}^{t_1} u(\tau) \cdot B(\tau)^T \phi(\tau) d\tau.
\end{align*}
\end{Korollar}

\begin{Beweis}
Vorbemerkung: Um die Aussage zu beweisen, wollen wir uns zunächst klarmachen, dass mit dem Hauptsatz der Differential- und Integralrechnung gilt:
\begin{align*}
\int_{t_0}^{t_1} \frac{d}{d\tau} (x(\tau) \cdot \phi(\tau)) d\tau 
&= x(t_1) \cdot  \underbrace{\phi(t_1)}_{=\phi_1} - \underbrace{x(t_0)}_{=0} \cdot \phi(t_0) \\
&= x(t_1) \cdot \phi_1
\end{align*}
Dann ergibt sich Folgendes:
\begin{align*}
x(t_1) \cdot \phi_1 &= \int_{t_0}^{t_1} \frac{d}{d\tau} (x(\tau) \cdot \phi(\tau)) d\tau\\
&=\int_{t_0}^{t_1} \left( \frac{d}{d\tau} x(\tau) \cdot \phi(\tau) + x(\tau) \cdot \frac{d}{d\tau} \phi(\tau)\right) d\tau \\
&=\int_{t_0}^{t_1}( \underbrace{(A(\tau)x(\tau) + B(\tau)u(\tau))}_{= \dot x(\tau)} \cdot \phi(\tau) - x(\tau) \cdot \underbrace{A(\tau)^T \phi(\tau)}_{=\dot \phi(\tau)} ) d\tau\\
&=\int_{t_0}^{t_1}\left( A(\tau)x(\tau) \cdot \phi(\tau)\right)+\left(B(\tau)u(\tau) \cdot \phi(\tau) \right) - \left( x(\tau) \cdot A(\tau)^T \phi(\tau) \right) d\tau\\
&= \int_{t_0}^{t_1}\left( [A(\tau)x(\tau) ]^T \phi(\tau)\right)+\left([B(\tau)u(\tau)]^T \phi(\tau) \right) - \left( x(\tau)^T A(\tau)^T \phi(\tau) \right) d\tau\\
&=\int_{t_0}^{t_1}\left( x(\tau)^TA(\tau)^T \phi(\tau)\right)+\left(u(\tau)^TB(\tau)^T \phi(\tau) \right) - \left( x(\tau)^T A(\tau)^T \phi(\tau) \right) d\tau\\
&=\int_{t_0}^{t_1}\left(u(\tau)^TB(\tau)^T \phi(\tau) \right)d\tau \\
&=\int_{t_0}^{t_1}\left(u(\tau) \cdot B(\tau)^T \phi(\tau) \right)d\tau
\end{align*}
Dies war unsere Behauptung.
(Bemerkung: Wobei $a \cdot b := a^Tb$, also das Standardskalarprodukt ist) 
\end{Beweis}

Wir wollen jetzt die HUM als Operatorfunktion $\Lambda$ formulieren:
\begin{Definition}
Wir definieren folgende Abbildung:
\begin{align*}
\Lambda : \mathbb R^n \rightarrow \mathbb R^n, \phi_1 \mapsto x(t_1)
\end{align*}
Also bildet $\Lambda$ einen Startzustand von $\phi$ auf einen Endzustand von $x$ ab.
Wobei $x: [t_0,t_1] \rightarrow \mathbb R^n$ die Lösung des Anfangswertproblems
\begin{align*}
\dot x = A(t)x + B(t) \bar u(t) ~~~~ x(t_0)=0
\end{align*}
mit
\begin{align*}
\bar u(t):=B(t)^T \phi(t)
\label{baru} \tag{$\star\star$}
\end{align*}
und $\phi :[t_0,t_1] \rightarrow \mathbb R^n$ Lösung des adjungierten Systems
\begin{align*}
\dot \phi = -A(t)^T \phi ~~~~ \phi(t_1) = \phi_1 , ~ t\in [t_0,t_1]
\end{align*}
 für ein beliebiges $\phi_1$ ist.
\end{Definition}

\begin{Korollar}
$\Lambda$ liefert uns also einen Algorithmus, um eine Steuerung $\bar u$ zu generieren, der wie folgt zu verstehen ist:
\begin{enumerate}
\item Wähle ein $\phi_1 \in \mathbb R^n$ 
\item Löse das Anfangswertproblem $\dot \phi = -A(t)^T \phi, ~ \phi(t_1) = \phi_1$
\item Berechne $\bar u(t)=B(t)^T \phi(t)$, wobei $\phi(t)$ die Lösung des in Schritt 2 gelösten Anfangswertproblems ist.
\item Löse das Anfangswertproblem $\dot x = A(t)x + B(t) \bar u(t), ~x(t_0)=0$
\end{enumerate}
\end{Korollar}

\begin{Bemerkung}[Eindeutigkeit]
Die Abbildung $\Lambda$ ist injektiv, da sowohl in Schritt 2, also auch in Schritt 4 eindeutig lösbare Anfangswertprobleme berechnet werden, somit ist $\bar u$ eindeutig bestimmt und damit wiederum wird jedem $\phi_1$ genau ein $x(T_1)$ zugeordnet.
\end{Bemerkung}

Was die Bemerkung bereits andeutet und den Vorteil der HUM ausmacht, fasst das folgende Theorem zusammen:
\begin{Satz}[Teil 1]
Mit den Vorausetzungen der vorangegangen Definition gilt:
\begin{align*}
\mathcal R = \Lambda(\mathbb R^n)
\end{align*}
Das heißt also, dass das Bild der Abbildung $\Lambda$ genau gleich der Erreichbarkeitsmenge $\mathcal R$ des Kontrollsystems (\ref{hum1}) ist.


\end{Satz}


\begin{Beweis}

"$\supseteq$":
Dies folgt direkt aus der Definition von $\Lambda$, da $\bar u \in L^{\infty}$ 
\\
"$\subseteq$":
Zu zeigen ist, dass ein beliebiges $x_1 \in \mathcal R$ im Bild der Abbildung $\Lambda$ liegt.
Sei also $x_1 \in \mathcal R$ beliebig, das heißt es existiert eine Sterung $u^*$, so dass die Lösung $x^*(t)$ sowohl das Anfangswertproblem $$\dot x = A(t)x + B(t)  u^*(t), ~x(t_0)=0$$
als auch $$x^*(t_1)=x_1$$ erfüllt.

Sei jetzt $\mathcal U \subset C^0([T_0,T_1], \mathbb R^m)$ definiert durch: $$ \mathcal U := \{ u \in C^0([T_0,T_1], \mathbb R^m) | ~ u(t) = B(t)^T\phi(t) \} $$
mit $\phi(t)$ Lösung des adjungierten Systems $\dot \phi = -A(t)^T \phi, ~ \phi(t_1) = \bar\phi $ für ein beliebiges $\bar\phi \in \mathbb R^n$.
Um später eine orthogonale Projektion auf $\mathcal U$ definieren zu können, müssen wir zeigen, dass $\mathcal U$ ein endlich-dimensionaler Untervektorraum (UVR) ist:
\begin{Lemma}
$\mathcal U$ ist ein Untervektorraum.
\end{Lemma}
\begin{Beweis}
\begin{enumerate}
\item[ ]
\item $0 \in \mathcal U$, da für $\phi_1=0$ gilt: $\phi(t) \equiv 0$, damit ist $\bar u(t) =B(t) \phi(t) \equiv 0$
\item Seien $u_1,u_2 \in \mathcal U$, dann ist zu zeigen, dass $u_1+u_2$ ebenfalls in $\mathcal U$ liegt:
\begin{align*}
u_1+u_2 = B(t)^T \phi_1(t) + B(t)^T\phi_2(t) = B(t)^T  \underbrace{(\phi_1(t) +\phi_2(t))}_{=\phi(t)\text{, da Lsg. VR}} \in \mathcal U
\end{align*}
\item Für ein $u\in \mathcal U$ und ein $\lambda \in \mathbb R$ gilt $\lambda u \in \mathcal U$.
\end{enumerate}
Damit ist $\mathcal U$ ein Untervektorraum von $C^0([T_0,T_1], \mathbb R^m)$ mit $dim(\mathcal U) \leq n$.
\end{Beweis}
Nachdem wir den Hilfssatz bewiesen haben, jetzt weiter zum unsprünglichen Beweis unseres Theorems:
Sei also $\bar u$ die orthogonale Projektion von $u^*$ auf $\mathcal U$.

\begin{Exkurs*}

Orthogonale Projektion: Sei $V$ ein endlich dimensionaler Vektorraum mit positiv definitem Skalarprodukt und $U$ ein Untervektorraum von $V$, dann ist die orthogonale Projektion auf $U$ eine Abbildung $p : V \rightarrow V$ mit:

Für alle $v \in V$ gilt
\begin{enumerate}
\item $p(v) \in U$
\item $v-p(v) \in U^{\perp}$
\end{enumerate}
\end{Exkurs*}
  
Dann gilt:
\begin{align*}
\int \underbrace{(u^*-\bar u) \cdot u}_{=0\text{, da ortho.}} = 0, ~~~~ \forall u \in \mathcal U
\end{align*}
$$\Leftrightarrow$$
\begin{align*}
\int u^* \cdot u = \int \bar u \cdot u,~~~~ \forall u \in \mathcal U
\end{align*}

Sei weiter $\bar x : [t_0,t_1]\leftarrow \mathbb R^n$ Lösung des Anfangswertproblems 
\[
\dot x = A(t)x + B(t) \underbrace{\cdot \bar u(t),}_{\text{orth. Proj.}} ~~~ x(t_0)=0
\]
 Dann gilt:
\begin{align*}
x_1 \cdot \phi_1 = x^*(t_1) \cdot \phi_1 &= \int_{t_0}^{t_1} u^* \cdot B(t)^T \phi(t) dt \\
&=\int_{t_0}^{t_1} \bar u \cdot B(t)^T \phi(t) dt \\
&=\bar x(t_1) \cdot \phi_1
\end{align*}
$\Rightarrow x_1 \cdot \phi_1 = \bar x(t_1) \cdot \phi_1 \Rightarrow \bar x(t_1) = x_1$

Da $\bar u \in \mathcal U$ existiert ein $\bar \phi_1 \in \mathbb R^n$, so dass $\bar \phi (t)$ Lösung des Anfangswertproblems  $\dot \phi = -A(t)^T \phi, ~ \phi(t_1) = \bar\phi_1 $. $\Lambda$ ist injektiv, damit folgt direkt  $x_1 = \Lambda(\bar\phi_1) \in \Lambda(\mathbb R^n)$ und somit ist die Inklusion gezeigt.

Und damit ist der erste Teil des Theorems gezeigt. Es gilt also $$\mathcal R = \Lambda(\mathbb R^n)$$
\end{Beweis}
Das heißt wir haben eine Möglichkeit gefunden einem jeden Punkt des $\mathbb R^n$ genau einen Punkt der Erreichbarkeitsmenge zuzuordnen. Zudem lässt sich die Steuerung $\bar u$ leichter berechnen und ist unabhängig von der Gramschen Kontrollmatix. Jetzt wollen wir noch zeigen, dass unser $\bar u$ die selbe Steuerung wie in Beweis zu \ref{Kontrollierbarkeit} ist, jedoch in einem allgemeinerem Rahmen, nämlich ohne Regularität von $\mathfrak C$:

Bereits in Beweis zu \ref{Kontrollierbarkeit} hatten wir gesehen, dass unsere Steuerung in gewissen Sinne optimal ist und dies sogar als einzige Steuerung. Also müssen wir nur zeigen, dass unsere, durch die HUM generierte Steuerung ebenfalls optimal  ist:
\begin{Satz}[Teil 2]
Wenn $x_1= \Lambda(\phi_1)$ und $u$ eine weitere Kontrollfunktion ist, welche das Kontrollsystem $\dot x = A(t)x+B(t)u,~ t\in[t_0,t_1]$ von $x_0=0$ nach $x_1 \in \mathbb R^n$ während des Zeitintervalls $[t_0,t_1]$ steuert. Dann gilt:
\begin{align*}
\int_{t_0}^{t_1} \|\bar u(\tau) \|^2 d\tau \leq \int_{t_0}^{t_1} \| u(\tau) \|^2 d\tau
\end{align*}
Wobei $\bar u$ wie in (\ref{baru}) und Gleichheit genau dann gilt, wenn $ u \equiv \bar u$.
\end{Satz}

\begin{Beweis}
Sei $v:=u-\bar u$ und damit $u=\bar u + v$, dann gilt wegen der Orthogonalität:
\begin{align*}
\int_{t_0}^{t_1} \| u(\tau) \|^2 d\tau = \int_{t_0}^{t_1} \|\bar u(\tau) \|^2 d\tau + \underbrace{\int_{t_0}^{t_1} \|v(\tau) \|^2 d\tau}_{\geq 0} + 2\cdot \underbrace{\int_{t_0}^{t_1} \| \bar u(\tau)\cdot v\|^2 d\tau}_{=0 \text{, orth. Proj.}}
\end{align*}
Insbesondere gilt $v\equiv 0$ nur, genau dann wenn $u \equiv \bar u$ ist.

Unser $\bar u$ ist also optimal und erfüllt damit die Gleichheit zur Steuerung aus Beweis zu \ref{Kontrollierbarkeit}.
\end{Beweis}
