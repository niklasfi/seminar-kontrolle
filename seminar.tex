\documentclass[a4paper]{article}


\usepackage[utf8]{inputenc}
\usepackage[T1]{fontenc}
\usepackage{cmbright} %cmbright typeface
\usepackage{amsmath}
\usepackage{enumitem}
\usepackage{amssymb}
\usepackage{eufrak} %gibt uns das wunderschöne C!
\usepackage{framed}
\usepackage[framed]{ntheorem}
\usepackage[normalem]{ulem}
\usepackage[pdfborder={1 1 0}]{hyperref}
\usepackage{fancyhdr}
\usepackage{todonotes}
\usepackage[ngerman]{babel}
\usepackage{makeidx}
\usepackage{biblatex}
\bibliography{seminar}
\setlength{\headheight}{12.5pt}

\fancypagestyle{mystyle}{ %
	\fancyhf{} % remove everything
	\renewcommand{\headrulewidth}{0.25pt} % remove lines as well
	\renewcommand{\footrulewidth}{0.25pt}
	\fancyhead[LE,RO]{\thepage}
	\fancyhead[RE,LO]{\leftmark}
}
%TODO spacing sollte genauso wie von section sein.
\makeatletter
\newtheoremstyle{newBreak}%
  {\item[\rlap{\vbox{\hbox{\hskip\labelsep \theorem@headerfont
          \large ##1\ ##2\theorem@separator}\hbox{\strut}}}]}%
  {\item[\rlap{\vbox{\hbox{\hskip\labelsep \theorem@headerfont
          \large ##1\ ##2\ (##3)\theorem@separator}\hbox{\strut}}}]}
\makeatother

\theorembodyfont{}
\theoremstyle{newBreak}

\newcounter{thmc}[section] %theorem counter
\renewcommand{\thethmc}{\thesection.\arabic{thmc}}
\newframedtheorem{Definition}[thmc]{Definition}
\newtheorem{Bemerkung}[thmc]{Bemerkung}
\newtheorem{Beispiel}[thmc]{Beispiel}
\theoremsymbol{\rule{1ex}{1ex}}
\newframedtheorem{Satz}[thmc]{Satz}
\newframedtheorem{Lemma}[thmc]{Lemma}
\newframedtheorem{Korollar}[thmc]{Korollar}

\theoremstyle{nonumberplain}
\theoremsymbol{\rule{1ex}{1ex}}
\newtheorem{Beweis}{Beweis}

\newcommand{\N}[0]{\mathbb{N}}
\newcommand{\Z}[0]{\mathbb{Z}}
\newcommand{\Q}[0]{\mathbb{Q}}
\newcommand{\R}[0]{\mathbb{R}}
\newcommand{\C}[0]{\mathbb{C}}
\newcommand{\G}[0]{\mathfrak{C}}

\newcommand{\plainset}[1]{\left\{#1\right\}}
\newcommand{\ouptoset}[1]{\plainset{1, ..., #1}} %one up to set
\newcommand{\zuptoset}[1]{\plainset{0, ..., #1}} %zero up to set
\newcommand{\condset}[2]{\left\{ #1: \; #2\right\}} %condition set

\renewcommand{\Re}[0]{\operatorname{Re}}
\renewcommand{\Im}[0]{\operatorname{Im}}
\newcommand{\sgn}[0]{\operatorname{Sgn}}
\newcommand{\argmin}[0]{\operatorname{Argmin}}
\newcommand{\argmax}[0]{\operatorname{Argmax}}
\newcommand{\dist}[0]{\operatorname{Dist}}

\newenvironment{enumerateRef}{\begin{enumerate}[ref={\thethmc\alph*},label={(\alph*) }, leftmargin=*]}{\end{enumerate}}
\newenvironment{enumerateBew}{\mbox{}\begin{enumerate}[label={zu (\alph*):}, leftmargin=*]}{\end{enumerate}}

\makeindex

\begin{document}
	\title{Seminararbeit zur Kontrolle}
	\author{Elisa Friebel, Marcel Goesmann, Niklas Fischer}	
	\date{\today}
	\maketitle
	\begin{abstract}
		Diese von Michel Herty betreute Arbeit ist eine Ausarbeitung des Kapitels "`Finite-dimensional linear control 
		systems"' aus dem Buch "`Control and Nonlinearity"' von Jean-Michel Coron \cite{Coron2007}.
	\end{abstract}
	\tableofcontents
	\newpage
	\pagestyle{mystyle}
\section{Definitionen}
	\begin{Bemerkung}
		In der ganzen Arbeit gelten die folgenden Konventionen:
		\begin{enumerate}
			\item $T_0$ und $T_1$ seien reelle Zahlen und es sei $T_0 < T_1$.
			\item $A: (T_0, T_1) \rightarrow \R^{n \times n}$ und $B: (T_0, T_1) 
				\rightarrow \R^{n \times m}$ seien stetige Funktionen.
			\item $x: (T_0, T_1) \rightarrow \R^n$ und $u: (T_0, T_1) 
				\rightarrow \R^m$ seien stetige Funktionen
		\end{enumerate}
	\end{Bemerkung}
	
	Kehren wir zunächst zurück zu einer altbekannten Definition:
	
	\begin{Definition}[Anfangswertproblem]\label{AWP}\index{Anfangswertproblem}\index{AWP}
	Sei $b: (T_0, T_1) \rightarrow \R^n$ eine stetige Funktion. Eine Funktion 
	$x$ wird Lösung des \emph{Anfangswertproblems (auch: AWP)} \[
		\dot x(t) = A(t) x(t) + b(t)
	\] genannt, falls diese Gleichung für alle $t \in [T_0, T_1]$ erfüllt ist, und
	für gegebene $x_0 \in \R^n$ und $t_0 \in [T_0, T_1]$ die Gleichung $x(t_0) = 
	x_0$ gilt.
	\end{Definition}
	
	Nun können wir lineare Kontrollsysteme definieren. Diese werden einen zentralen Punkt in der Arbeit einnehmen. 
	
	\begin{Definition}[Lineares Kontrollsystem]\index{lineares Kontrollsystem}
		Ein System
		\begin{align*}%\label{gl:ks}
			\dot x(t) = A(t) x(t) + B(t) u(t),\quad t \in [T_0, T_1]
		\end{align*}
		wird \emph{lineares Kontrollsystem} genannt.
	\end{Definition}
	
		Anschaulich könnte man lineare Kontrollsysteme so interpretieren, dass zum 
		Beispiel ein Raumfahrzeug mit einer Steuerung $u$ vom Startpunkt $x_0$ zu 
		einem Punkt $x_1$ gebracht werden soll. Dabei modelliert $A$ die 
		"`natürlichen"' Kräfte auf das Objekt, und $B$ bildet Eigenschaften der 
		Steuerungseinheit des Raumschiffes, beispielsweise die Anordnung der 
		Triebwerke, ab. $u$ wiederum stellt die konkrete Steuerung dar, also zu 
		welchem Zeitpunkt die Triebwerke wie angesteuert werden sollen.
	
	\begin{Definition}[kontrollierbar]\index{kontrollierbar}
		Seien $x_0, x_1 \in \R^n$. Ein lineares Kontrollsystem 
			\[
				\dot x(t) = A(t)x(t) + B(t)u(t)
			\]
		ist \emph{kontrollierbar}, falls es eine stetige Funktion $u$ gibt, sodass die Lösungfunktion $x$ des Anfangswertproblems 
		\begin{align*}
			\dot x(t) = A(t)x(t) + B(t)u(t) \; \forall t \in [T_0, T_1] \quad \text{mit} \quad x(T_0) = x_0
		\end{align*}
		in $T_1$ den Wert $x_1$ annimmt.
	\end{Definition}
	Im Gegensatz zu dem oben beschriebenen Anwendungsfall ist für die 
	Kontrollierbarkeit eines Systems also nicht nur erforderlich, dass die Sonde 
	von der Erde aus einen bestimmten Punkt im Weltall erreichen kann, sondern, 
	dass sie von jedem Punkt ausgehend jeden anderen Erreichen kann.
	
	
	\section{Die Resolvente}
	\subsection*{Einleitung}
		Von besonderem Interesse ist jetzt natürlich ein Kriterium für die 
		Kontrolleribarkeit eines linearen Kontrollsystems. Dieses wird in
		Abschnitt \ref{Gram'sche Kontrollmatrix} geliefert. Hierfür ist allerdings
		etwas Vorarbeit notwendig.
		
		Für lineare Differentialgleichungssysteme \[
			\dot x(t) = A \cdot x(t)
		\] mit konstanten $A \in \R ^ {n \times n}$ ist es mithilfe der 
		Exponentialfunktion einfach eine Fundamentalmatrix zu bestimmen, doch was
		geschieht, wenn $A$ wie in der in \ref{AWP} gegebenen Definition eines 
		Anfangswertproblems abhängig von $t$ ist? In diesem Fall ist die
		Fundamentalmatrix im allgemeinen nicht elementar bestimmbar.
		
		Diese Klippe wird mit Resolvente $R$, die uns eine eindeutig bestimmte 
		Fundamentalmatrix eines Anfangswertproblems	liefert, geschickt umschifft.
		Mit der Resolvente und dem zentralen Satz aus Abschnitt
		\ref{Gram'sche Kontrollmatrix} können kann dann überpüft werden, ob das 
		lineare Kontrollproblem kontrollierbar ist. 
	\subsection*{Definition}
		Sei $\Phi$ die Funktion, die das homogene Anfangswertproblem
		\[
			\dot M(t) = A(t)\cdot M(t) \; \forall t \in [T_0, T_1]  \quad \text{mit} \quad M(\tau) = E_n
		\]
		löst. Offensichtlich ist $\Phi$ von der Wahl von $\tau$ abhängig. Definiere
		also $\tilde R$ als die Funktion, die $\tau$ auf die Funktion $\Phi$ 
		abbildet.
		
		\[
			\tilde R : [T_0, T_1] \rightarrow C^0( [T_0, T_1],\R^{n \times n}), 
			\tau \mapsto \Phi
		\]
	
		Jetzt kann die Resolvente einfach definiert werden:
		
		\begin{Definition}[Resolvente]
			Die Funktion 
			\[
				R: [T_0, T_1] \times [T_0, T_1] \rightarrow \R^{n \times n},
				(t, \tau) \mapsto \tilde R(\tau)(t)
			\]
			heißt Resolvente.
		\end{Definition}
		
	\subsection*{Eigenschaften}
		\begin{Lemma}
			$R$ ist stetig.
		\end{Lemma}
		\begin{Beweis}
			Die Stetigkeit von $R$ in der 1. Komponente ($t$) folgt aus der Stetigkeit
			jeder Lösungsfunktion $\Phi$ des von $\tau$ abhängigen Anfangswertproblems.
			
			Die Stetigkeit in der 2. Komponente ($\tau$) folgt aus dem Satz über die
			Stetige Abhängigkeit aus \cite{KriegWalcher2010}.
			
			Betrachte die Anfangswertprobleme
			\[
				\dot M(t) = A(t)\cdot M(t) \; \forall t \in [T_0, T_1]  \quad \text{mit} \quad M(\tau_1) = E_n
			\]
			und
			\[
				\dot M(t) = A(t)\cdot M(t) \; \forall t \in [T_0, T_1]  \quad \text{mit} \quad M(\tau_2) = E_n
			\]
			mit Lösungen $\Phi_1$ bzw. $\Phi_2$.
			Dann existiert zu jedem $\varepsilon > 0 $ ein $\delta := | \tau_1 - \tau_2 |$
			sodass \[
				|| \Phi_1(t) - \Phi_2(t) || < \varepsilon \; \forall t \in [T_0, T_1]
			\]
			dies ist äquivalent zur Stetigkeit von $R$ in der 2. Komponente.
			Da $R$ in beiden seiner Komponenten stetig ist, it $R$ stetig.
		\end{Beweis}
		
		\begin{Lemma}\label{Resolvente Eigenschaften}
			Die Resolvente erfüllt die folgenden Eigenschaften
			\begin{enumerateRef}
				\item\label{Resolvente Eigenschaften:1} $	R(t_1,t_1) = E_n$
				\item\label{Resolvente Eigenschaften:2} $ R(t_1, t_2) \cdot R(t_2, t_3) = R(t_1, t_3)$
				\item\label{Resolvente Eigenschaften:3} $ R(t_1, t_2) \cdot R(t_2, t_1) = E_n$
			\end{enumerateRef}
			für alle $t_1, t_2, t_3 \in [T_0, T_1]$.
		\end{Lemma}
		
		\begin{Beweis}
		\begin{enumerateBew}
			\item Betrachten wir das zu $R(t_1, t_1) $ gehörige 
			Anfangswertproblem \[
				\dot M(t) = A(t) \cdot M(t) \; \forall t \in [T_0, T_1] \quad \text{mit} \quad M(t_1) = E_n
			\] mit Lösung $\Phi$.
			Nun gilt: \[
				R(t_1, t_1) = \tilde R (t_1)(t_1) = \Phi(t_1) = E_n
			\]
			\item \todo{BEWEIS???}
			\item 
				\[
					R(t_1, t_2) \cdot R(t_2, t_1) \overset{2.}= R(t_1, t_1) \overset{1.}= E_n
				\]
		\end{enumerateBew}
		\end{Beweis}
		
		\begin{Lemma}\label{Resolvente Ableitung}
			Für die Ableitungen der Resolvente gilt:
			\begin{enumerateRef}
				\item\label{Resolvente Ableitung:1} $ \frac{\partial R}{\partial t} (t, \tau) = A(t) \cdot R(t,\tau) \; \forall t,\tau \in [T_0, T_1]$
				\item\label{Resolvente Ableitung:2} $ \frac{\partial R}{\partial \tau} (t, \tau) = - R(t, \tau) \cdot A(\tau) \; \forall t,\tau \in [T_0, T_1]$
			\end{enumerateRef}
		\end{Lemma}
		
		\begin{Beweis}
			\begin{enumerateBew}
				\item
					Die Definition der Resolvente mit ihrem Anfangswertproblem gibt uns
					die Interpretation \[
						R(t, \tau ) = E_n + \int_\tau^t A(t) \cdot R(t, \tau) dT
					\]
					Differenzieren liefert:
					\[
						\frac{\partial R}{\partial t}R(t,\tau) = A(t) \cdot R(t, \tau)
					\]
				\item
					Differenzieren von \ref{Resolvente Eigenschaften:3} nach $t_2$ gibt uns
					\[
						\frac{\partial R}{\partial t_2} E_n = \frac{\partial R}{\partial t_2} R(t_1, t_2) \cdot R(t_2, t_1)
					\] also \[
						0 = \left[ \frac{\partial R}{\partial t_2} R(t_1, t_2) \right] \cdot R(t_2, t_1) +
							R(t_1, t_2) \cdot \underbrace{\left[ \frac{\partial R}{\partial t_2}R(t_2, t_1) \right]}_{=A(t_2) \cdot R(t_2, t_1)} 
					\] woraus \[
						0 = \left [ \frac{\partial R}{\partial t_1}R(t_1, t_2) + R(t_1, t_2) \cdot A(t_2) \right] \cdot R(t_2, t_1)
					\] folgt.
					
					\ref{Resolvente Eigenschaften:3} liefert uns, dass $R(t_1, t_2)$ 
					invertierbar ist ($R(t_2, t_1)$ ist die zugehörige Inverse). Also
					gilt:
					\[
						\frac{\partial R}{\partial t_2}R(t_1, t_2) = - R(t_1, t_2) \cdot A(t_2)
					\]
			\end{enumerateBew}
		\end{Beweis}
	
		\begin{Lemma}\label{Resolvente AWP}
			Für die Lösung $x$ des Anfangswertproblems \[
				\dot x(t) = A(t) \cdot x(t) + b(t) \; \forall t \in [T_0, T_1] \quad \text{mit} \quad x(T_0) = x^0
			\] gilt 
			\begin{enumerateRef}
				\item\label{Resolvente AWP:1} $x(t_1) = R(t_1, t_0) \cdot x(t_0) + \int_{t_0}^{t_1} R(t_1,\tau)\cdot b(\tau) d\tau$
				\item\label{Resolvente AWP:2} $x(t) = R(t, T_0) \cdot x^0+ \int_{T_0}^t R(t, \tau) \cdot b(\tau)d\tau$
			\end{enumerateRef} für alle $t_0, t_1, t \in [T_0, T_1]$
		\end{Lemma}

		\begin{Beweis}
			\begin{enumerateBew}
				\item
					 \todo{fehlt!}
				\item
					Spezialfall von \ref{Resolvente AWP:1} mit $t_0:= T_0$ und $t_1 := t$ 
			\end{enumerateBew}
		\end{Beweis}
		
	\section{Gram'sche Kontrollmatrix}\label{Gram'sche Kontrollmatrix}
		\begin{Definition}[Gram'sche Kontrollmatrix]
			Für ein Kontrollsystem \[
				\dot x(t) = A(t) \cdot x(t) + B(t) \cdot u(t)
			\]
			heißt die Matrix \[
				\G:=\int_{T_0}^{T_1} R(T_1, \tau) \cdot B(\tau) \cdot B(\tau)^T\cdot R(T_1, \tau)^T d\tau \in \R^{n \times n}
			\] Gram'sche Kontrollmatrix
		\end{Definition}
		
		\begin{Lemma}
			$\G$ ist symmetrisch positiv semidefinit.
		\end{Lemma}
		
		\begin{Beweis}
			\todo{fehlt!}
		\end{Beweis}
		
		\begin{Satz}
			Ein lineares Kontrollsystem \[
				\dot x(t) = A(t) \cdot x(t) + B(t) \cdot u(t)
			\] ist genau dann kontrollierbar, wenn seine
			Gram'sche Kontrollmatrix regulär ist.
		\end{Satz}
		
		\begin{Beweis}
			\begin{enumerate}
				\item["`$\Leftarrow$"': ]
					Sei $\G$ invertierbar, $x^0, x^1 \in R^n$ beiliebig. Definiere \[
						\bar u: (T_0, T_1) \rightarrow R^m, \tau \mapsto B(\tau)^T\cdot R(T_1,\tau)^T\G^{-1}\cdot (x^1-R(T_1, T_0)\cdot x^0)
					\]
					Dann ist $\bar u$ stetig. \todo{Wieso?}
					
					Sei $\bar x$ die Lösung des Anfangswertproblems \[
						\dot {\bar x}(t) = A(t) \cdot \bar x(t) + B(t) \cdot \bar u (t) \quad \text{mit} \quad \bar x (T_0) = x^0
					\]
					
					Nach \ref{Resolvente AWP:2} gilt: \begin{align*}
						\bar x(T_1) = & R(T_1, T_0) \cdot x^0 + \int_{T_0}^{T_1} R(T_1, \tau ) \cdot B(\tau) \cdot \bar u(\tau) d\tau \\
						 = & R(T_1, T_0) x^0 + \\
						 & \qquad \int_{T_0}^{T_1} R(T_1, \tau ) B(\tau) B(\tau)^T R(T_1,\tau)^T\G^{-1}(x^1-R(T_1, T_0)x^0) d\tau\\
						 = & R(T_1, T_0) x^0 + \\
						 & \qquad \underbrace{\int_{T_0}^{T_1} R(T_1, \tau ) B(\tau) B(\tau)^T R(T_1,\tau)^T d\tau}_{=\G} \; \G^{-1}(x^1-R(T_1, T_0)x^0) \\
						 = & R(T_1, T_0) x^0 + x^1-R(T_1, T_0)x^0 \\
						 = & x^1
					\end{align*}
					
					also gibt es eine Steuerung $\bar u$, die die gewünschten Bedingungen erfüllt.
					
				\item["`$\Rightarrow$"': ]
					Falls $\G$ singulär ist, gibt es ein $y \in \R^n\setminus \plainset{0}$ mit \[
						\G y = 0
					\]
					Also gilt auch
					\begin{align*}
						0 = y^T \G y = &\int_{T_0}^{T_1} y^T R(T_1,y)B(\tau)B(\tau)^T R(T_1, \tau)^T y d\tau \\
						= & \int_{T_0}^{T_1} | B(\tau)^T R(T_1, \tau)^T y |^2 d\tau
					\end{align*}
					Daraus folgt 
					\[ \tag{$\star$}\label{eqn:Gramsche Kontrollmatrix:star}
						y^T R(T_1, \tau) B(\tau) = 0
					\] da $\tau \mapsto B(\tau)^T R(T_1, \tau)^T y$ stetig ist.
					Sei nun $u: [T_0, T_1] \rightarrow R^m$ eine beliebige stetige Funktion. Dazu sei $x$ die Lösung des Anfangswertproblems \[
						\dot x(t) = A(t) x(t) + B(t) u(t) \quad \text{mit} \quad x(T_1) = 0 = x^1
					\]
					Dann gilt wieder mit \ref{Resolvente AWP:2} \begin{align*}
						x(t) = & R(t, T_0) \cdot x^0+ \int_{T_0}^t R(t, \tau) \cdot B(\tau) u(\tau) d\tau \\
								 = & \int_{T_0}^t R(t, \tau) \cdot B(\tau) u(\tau) d\tau \tag{$\star\star$}\label{eqn:Gramsche Kontrollmatrix:star2}
					\end{align*}
					
					Aus \eqref{eqn:Gramsche Kontrollmatrix:star} und \eqref{eqn:Gramsche Kontrollmatrix:star2} folgt
					\begin{align*}
						y^T x(t) = & y^T \int_{T_0}^t R(t, \tau) \cdot B(\tau) u(\tau) d\tau \\
						= & \int_{T_0}^t \underbrace{ y^T R(t, \tau) \cdot B(\tau)}_{=0} u(\tau) d\tau = 0
					\end{align*}
					also, dass $ y^T x(T_1) = 0$ für beliebige $u$ gilt.
					
					Es existiert aber ein $x^1 \in \R^n$ mit $y^T x^1 \neq 0$, beispielsweise $x^1 = y$.
					
					Wenn das Kontrollproblem kontrollierbar wäre würde es eine Steuerung $u$ geben, sodass $x(T_1) = x^1$.
					Daraus würde dann aber auch \[0 = y^T x(T_1) = y^T x^1 \neq 0\] folgen, was ein Widerspruch ist.
					
					Also kann das Kontrollproblem nicht kontrollierbar sein, wenn $\G$ singulär ist.
			\end{enumerate}
		\end{Beweis}
		
		\begin{Satz}\label{Gramsche Minimalität}
			Seien $T_0, T_1\in \R$ sowie $x^0, x^1 \in \R^n$. Für ein beliebiges Kontrollproblem
			\[
				\dot x = A(t) x(t) + B(t) u(t) \; \forall t\in [T_0, T_1] \quad \text{mit} \quad x(T_0)=x^0, \; x(T_1)=x^1
			\]
			mit Lösung $u$ gilt
			\begin{enumerateRef}
				\item $\int_{T_0}^{T_1} |\bar u(\tau)|^2 d\tau < \int_{T_0}^{T_1} |u(\tau)|^2d\tau$, falls $u \neq \bar u$ und
				\item $\int_{T_0}^{T_1} |\bar u(\tau)|^2 d\tau = \int_{T_0}^{T_1} |u(\tau)|^2d\tau$, falls $u = \bar u$
			\end{enumerateRef}
			wobei
			\[
				\bar u: (T_0, T_1) \rightarrow R^m, \tau \mapsto B(\tau)^T\cdot R(T_1,\tau)^T\G^{-1}\cdot (x^1-R(T_1, T_0)\cdot x^0)
			\]
		\end{Satz}
		
		\begin{Beweis}
			Definiere \[
				v := u - \bar u
			\]
			Betrachte zunächst $|u(t)|^2$
			\begin{align*}
				|u(t)|^2 = & |\bar u(t) + v(t)|^2 \\
					= & \langle \bar u(t) + v(t), \bar u(t) + v(t)\rangle \\
					= & \langle \bar u(t), \bar u(t)\rangle + 2 \langle \bar u(t), v(t)\rangle + \langle v(t),v(t) \rangle \\
					= & |\bar u(t) |^2 + 2 \bar u(t)^T v(t) + |v(t)|^2
			\end{align*}
			Also gilt auch
			\[
				\int_{T_0}^{T_1}|u(\tau)d\tau|^2 = \int_{T_0}^{T_1}|\bar u(\tau) |^2 + 2 \int_{T_0}^{T_1}\bar u(\tau)^T v(\tau) d\tau+ \int_{T_0}^{T_1}|v(\tau)|^2 d\tau
			\]
		Für $\int_{T_0}^{T_1}\bar u(\tau)^T v(\tau) d\tau$ aber gilt
		\begin{align*}
			& \int_{T_0}^{T_1}\bar u(\tau)^T v(\tau) d\tau \\
			\overset{\text{Def. }u}= & \int_{T_0}^{T_1}(B(\tau)^T R(T_1,\tau)^T\G^{-1} (x^1-R(T_1, T_0) x^0))^T v(\tau) d\tau \\
			\overset{\G \text{ sym.}}= & \int_{T_0}^{T_1} (x^1-R(T_1,T_0)x^0)^T \G^{-1}R(T_1,\tau)B(\tau)v(\tau)d\tau \\
			= \;\;\,& (x^1-R(T_1,T_0)x^0)^T \G^{-1} \int_{T_0}^{T_1} R(T_1,\tau)B(\tau)v(\tau)d\tau
		\end{align*}
		und für $\int_{T_0}^{T_1} R(T_1,\tau)B(\tau)v(\tau)d\tau$ gilt 
		\begin{align*}
			 & \int_{T_0}^{T_1} R(T_1,\tau)B(\tau)v(\tau)d\tau \\
			=& \int_{T_0}^{T_1} R(T_1,\tau)B(\tau)u(\tau)d\tau - \int_{T_0}^{T_1} R(T_1,\tau)B(\tau)\bar u(\tau)d\tau \\
			\overset{\eqref{eqn:Gramsche Minimalitaet:star}}=& (x(T_1) - R(T_1,T_0)x(T_0)) - (\bar x(T_1)-R(T_1,T_0) \bar x(T_0)) \\
			\overset{{x, \bar x \text{ Lsg.}}}=& (x^1-R(T_1,T_0)x^0) - (x^1-R(T_1,T_0)x^0) \\
			=& 0
			\tag{$\star\star$}\label{eqn:Gramsche Minimalitaet:star2} 
		\end{align*}
		wobei \eqref{eqn:Gramsche Minimalitaet:star} aus mit Umstellen aus \ref{Resolvente AWP:2} folgt:
		\begin{align*}
			& x(T_1) =R(T_1,T_0) x^0 + \int^{T_1}_{T_0}R(T^1,\tau)b(\tau)d\tau \\
			\Leftrightarrow & x(T_1) - R(T_1,T_0) x^0 = \int^{T_1}_{T_0}R(T^1,\tau)b(\tau)d\tau
			\tag{$\star$}\label{eqn:Gramsche Minimalitaet:star} 
		\end{align*}
		Mit \eqref{eqn:Gramsche Minimalitaet:star2} gilt also
		\[
			\int_{T_0}^{T_1} R(T_1,\tau)B(\tau)v(\tau)d\tau = 0
		\]
		und damit
		\[
			\int_{T_0}^{T_1}|u(\tau)d\tau|^2 = \int_{T_0}^{T_1}|\bar u(\tau) |^2 +  \underbrace{\int_{T_0}^{T_1}|v(\tau)|^2 d\tau}_{\geq 0}
		\]
		\end{Beweis}
\newpage
\printbibliography
\newpage
\printindex
\end{document}

% ===REFERENZEN===
% Krieg - Gew. DGl
% Rannacher
