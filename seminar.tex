\documentclass[a4paper]{article}

\usepackage[utf8]{inputenc}
\usepackage[T1]{fontenc}
\usepackage{cmbright} %cmbright typeface
\usepackage{amsmath}
\usepackage{enumerate}
\usepackage{amssymb}
\usepackage{framed}
\usepackage[framed]{ntheorem}
\usepackage[normalem]{ulem}
\usepackage[pdfborder={1 1 0}]{hyperref}
\usepackage{fancyhdr}
\usepackage{todonotes}
\usepackage[ngerman]{babel}
\usepackage{makeidx}
\usepackage{biblatex}
\bibliography{seminar}
\setlength{\headheight}{12.5pt}

\fancypagestyle{mystyle}{ %
	\fancyhf{} % remove everything
	\renewcommand{\headrulewidth}{0.25pt} % remove lines as well
	\renewcommand{\footrulewidth}{0.25pt}
	\fancyhead[LE,RO]{\thepage}
	\fancyhead[RE,LO]{\leftmark}
}
%TODO spacing sollte genauso wie von section sein.
\makeatletter
\newtheoremstyle{newBreak}%
  {\item[\rlap{\vbox{\hbox{\hskip\labelsep \theorem@headerfont
          \large ##1\ ##2\theorem@separator}\hbox{\strut}}}]}%
  {\item[\rlap{\vbox{\hbox{\hskip\labelsep \theorem@headerfont
          \large ##1\ ##2\ (##3)\theorem@separator}\hbox{\strut}}}]}
\makeatother

\theorembodyfont{}
\theoremstyle{newBreak}

\newcounter{thmc}[section] %theorem counter
\renewcommand{\thethmc}{\thesection.\arabic{thmc}}
\newframedtheorem{Definition}[thmc]{Definition}
\newtheorem{Bemerkung}[thmc]{Bemerkung}
\newtheorem{Beispiel}[thmc]{Beispiel}
\theoremsymbol{\rule{1ex}{1ex}}
\newframedtheorem{Satz}[thmc]{Satz}
\newframedtheorem{Lemma}[thmc]{Lemma}

\theoremstyle{nonumberplain}
\theoremsymbol{\rule{1ex}{1ex}}
\newtheorem{Beweis}{Beweis}

\newcommand{\N}[0]{\mathbb{N}}
\newcommand{\Z}[0]{\mathbb{Z}}
\newcommand{\Q}[0]{\mathbb{Q}}
\newcommand{\R}[0]{\mathbb{R}}
\newcommand{\C}[0]{\mathbb{C}}

\newcommand{\plainset}[1]{\left\{#1\right\}}
\newcommand{\ouptoset}[1]{\plainset{1, ..., #1}} %one up to set
\newcommand{\zuptoset}[1]{\plainset{0, ..., #1}} %zero up to set
\newcommand{\condset}[2]{\left\{ #1: \; #2\right\}} %condition set

\renewcommand{\Re}[0]{\operatorname{Re}}
\renewcommand{\Im}[0]{\operatorname{Im}}
\newcommand{\sgn}[0]{\operatorname{Sgn}}
\newcommand{\argmin}[0]{\operatorname{Argmin}}
\newcommand{\argmax}[0]{\operatorname{Argmax}}
\newcommand{\dist}[0]{\operatorname{Dist}}

\makeindex

\begin{document}
	\title{Seminararbeit zur Kontrolle}
	\author{Elisa Friebel, Marcel Goesmann, Niklas Fischer, betreut durch: Michael Herty}	
	\date{\today}
	\maketitle
	\begin{abstract}
		-- Zusammsenfassung --
	\end{abstract}
	\tableofcontents
	\newpage
	\pagestyle{mystyle}
\section{Definitionen}
	\begin{Bemerkung}
		In der ganzen Arbeit gelten die folgenden Konventionen:
		\begin{enumerate}
			\item $T_0$ und $T_1$ seien reelle Zahlen und es sei $T_0 < T_1$.
			\item $A: (T_0, T_1) \rightarrow \R^{n \times n}$ und $B: (T_0, T_1) 
				\rightarrow \R^{n \times m}$ seien stetige Funktionen.
			\item $x: (T_0, T_1) \rightarrow \R^n$ und $u: (T_0, T_1) 
				\rightarrow \R^m$ seien stetige Funktionen
		\end{enumerate}
	\end{Bemerkung}
	
	Kehren wir zunächst zurück zu einer altbekannten Definition:
	
	\begin{Definition}[Anfangswertproblem]\label{AWP}\index{Anfangswertproblem}\index{AWP}
	Sei $b: (T_0, T_1) \rightarrow \R^n$ eine stetige Funktion. Eine Funktion 
	$x$ wird Lösung des \emph{Anfangswertproblems (auch: AWP)} \[
		\dot x(t) = A(t) x(t) + b(t)
	\] genannt, falls diese Gleichung für alle $t \in [T_0, T_1]$ erfüllt ist, und
	für gegebene $x_0 \in \R^n$ und $t_0 \in [T_0, T_1]$ die Gleichung $x(t_0) = 
	x_0$ gilt.
	\end{Definition}
	
	Nun können wir lineare Kontrollsysteme definieren. Diese werden einen zentralen Punkt in der Arbeit einnehmen. 
	
	\begin{Definition}[Lineares Kontrollsystem]\index{lineares Kontrollsystem}
		Ein System
		\begin{align*}%\label{gl:ks}
			\dot x(t) = A(t) x(t) + B(t) u(t),\quad t \in [T_0, T_1]
		\end{align*}
		wird \emph{lineares Kontrollsystem} genannt.
	\end{Definition}
	
		Anschaulich könnte man lineare Kontrollsysteme so interpretieren, dass zum 
		Beispiel ein Raumfahrzeug mit einer Steuerung $u$ vom Startpunkt $x_0$ zu 
		einem Punkt $x_1$ gebracht werden soll. Dabei modelliert $A$ die 
		"`natürlichen"' Kräfte auf das Objekt, und $B$ bildet Eigenschaften der 
		Steuerungseinheit des Raumschiffes, beispielsweise die Anordnung der 
		Triebwerke, ab. $u$ wiederum stellt die konkrete Steuerung dar, also zu 
		welchem Zeitpunkt die Triebwerke wie angesteuert werden sollen.
	
	\begin{Definition}[kontrollierbar]\index{kontrollierbar}
		Seien $x_0, x_1 \in \R^n$. Ein lineares Kontrollsystem 
			\[
				\dot x(t) = A(t)x(t) + B(t)u(t)
			\]
		ist \emph{kontrollierbar}, falls es eine stetige Funktion $u$ gibt, sodass die Lösungfunktion $x$ des Anfangswertproblems 
		\begin{align*}
			\dot x(t) = A(t)x(t) + B(t)u(t) \; \forall t \in [T_0, T_1] \quad \text{mit} \quad x(T_0) = x_0
		\end{align*}
		in $T_1$ den Wert $x_1$ annimmt.
	\end{Definition}
	Im Gegensatz zu dem oben beschriebenen Anwendungsfall ist für die 
	Kontrollierbarkeit eines Systems also nicht nur erforderlich, dass die Sonde 
	von der Erde aus einen bestimmten Punkt im Weltall erreichen kann, sondern, 
	dass sie von jedem Punkt ausgehend jeden anderen Erreichen kann.
	
	
	\section{Die Resolvente}
	\subsection*{Einleitung}
		Von besonderem Interesse ist jetzt natürlich ein Kriterium für die 
		Kontrolleribarkeit eines linearen Kontrollsystems. Dieses wird uns in
		Abschnitt \ref{Gram'sche Kontrollmatrix} geliefert. Hierfür ist allerdings
		etwas vorarbeit notwendig.
		
		Für lineare Differentialgleichungssysteme \[
			\dot x(t) = A \cdot x(t)
		\] mit konstanten $A \in \R ^ {n \times n}$ ist es mithilfe der 
		Exponentialfunktion einfach eine Fundamentalmatrix zu bestimmen, doch was
		geschieht, wenn $A$ wie in der in \ref{AWP} gegebenen Definition eines 
		Anfangswertproblems abhängig von $t$ ist? In diesem Fall ist die
		Fundamentalmatrix im allgemeinen nicht elementar bestimmbar.
		
		Diese Klippe umbschiffen wir geschickt mit der Definition der Resolvente $R$,
		die uns eine eindeutig bestimmte Fundamentalmatrix eines Anfangswertproblems
		liefert. Mit der Resolvente und dem zentralen Satz aus Abschnitt
		\ref{Gram'sche Kontrollmatrix} können wir dann überprüfen, ob das lineare 
		Kontrollproblem kontrollierbar ist. 
	\subsection*{Definition}
		Sei $\Phi$ die Funktion, die das homogene Anfangswertproblem
		\[
			\dot M(t) = A(t)\cdot M(t) \; \forall t \in [T_0, T_1]  \quad \text{mit} \quad M(\tau) = E_n
		\]
		löst. Offensichtlich ist $\Phi$ von der Wahl von $\tau$ abhängig. Definiere
		also $\tilde R$ als die Funktion, die $\tau$ auf die Funktion $\Phi$ 
		abbildet.
		
		\[
			\tilde R : [T_0, T_1] \rightarrow C^0( [T_0, T_1],\R^{n \times n}), 
			\tau \mapsto \Phi
		\]
	
		Jetzt können wir die Resolvente einfach definieren:
		
		\begin{Definition}[Resolvente]
			Die Funktion 
			\[
				R: [T_0, T_1] \times [T_0, T_1] \rightarrow \R^{n \times n},
				(t, \tau) \mapsto \tilde R(\tau)(t)
			\]
			heißt Resolvente.
		\end{Definition}
		
	\subsection*{Eigenschaften}
		Wir beginnen damit, uns die Eigenschaften der Resolvente vor Augen zu
		führen.
		\begin{Lemma}
			$R$ ist stetig.
		\end{Lemma}
		\begin{Beweis}
			Die Stetigkeit von $R$ in der 1. Komponente ($t$) folgt aus der Stetigkeit
			jeder Lösungsfunktion $\Phi$ des von $\tau$ abhängigen Anfangswertproblems.
			
			Die Stetigkeit in der 2. Komponente ($\tau$) folgt aus dem Satz über die
			Stetige Abhängigkeit aus \cite{KriegWalcher2010}.
			
			Betrachte die Anfangswertprobleme
			\[
				\dot M(t) = A(t)\cdot M(t) \; \forall t \in [T_0, T_1]  \quad \text{mit} \quad M(\tau_1) = E_n
			\]
			und
			\[
				\dot M(t) = A(t)\cdot M(t) \; \forall t \in [T_0, T_1]  \quad \text{mit} \quad M(\tau_2) = E_n
			\]
			mit Lösungen $\Phi_1$ bzw. $\Phi_2$.
			Dann existiert zu jedem $\varepsilon > 0 $ ein $\delta := | \tau_1 - \tau_2 |$
			sodass \[
				|| \Phi_1(t) - \Phi_2(t) || < \varepsilon \; \forall t \in [T_0, T_1]
			\]
			dies ist äquivalent zur Stetigkeit von $R$ in der 2. Komponente.
			Da $R$ in beiden seiner Komponenten stetig ist, it $R$ stetig.
		\end{Beweis}
		
		\begin{Lemma}\label{Resolvente Eigenschaften}
			Die Resolvente erfüllt die folgenden Eigenschaften
			\begin{enumerate}
				\item $	R(t_1,t_1) = E_n$
				\item $ R(t_1, t_2) \cdot R(t_2, t_3) = R(t_1, t_3)$
				\item $ R(t_1, t_2) \cdot R(t_2, t_1) = E_n$
			\end{enumerate}
			für alle $t_1, t_2, t_3 \in [T_0, T_1]$.
		\end{Lemma}
		
		\begin{Beweis}
		\begin{enumerate}
			\item[zu 1.] Betrachten wir das zu $R(t_1, t_1) $ gehörige 
			Anfangswertproblem \[
				\dot M(t) = A(t) \cdot M(t) \; \forall t \in [T_0, T_1] \quad \text{mit} \quad M(t_1) = E_n
			\] mit Lösung $\Phi$.
			Nun gilt: \[
				R(t_1, t_1) = \tilde R (t_1)(t_1) = \Phi(t_1) = E_n
			\]
			\item[zu 2.] \todo{BEWEIS???}
			\item[zu 3.]
				\[
					R(t_1, t_2) \cdot R(t_2, t_1) \overset{2.}= R(t_1, t_1) \overset{1.}= E_n
				\]
		\end{enumerate}
		\end{Beweis}
		
		\begin{Lemma}
			Für die Ableitungen der Resolvente gilt:
			\begin{enumerate}
				\item $ \frac{\partial R}{\partial t} (t, \tau) = A(t) \cdot R(t,\tau) \; \forall t,\tau \in [T_0, T_1]$
				\item $ \frac{\partial R}{\partial \tau} (t, \tau) = - R(t, \tau) \cdot A(\tau) \; \forall t,\tau \in [T_0, T_1]$
			\end{enumerate}
		\end{Lemma}
		
		\begin{Beweis}
			\begin{enumerate}
				\item[zu 1.]
					Die Definition der Resolvente mit ihrem Anfangswertproblem gibt uns
					die Interpretation \[
						R(t, \tau ) = E_n + \int_\tau^t A(t) \cdot R(t, \tau) dT
					\]
					Differenzieren liefert:
					\[
						\frac{\partial R}{\partial t}R(t,\tau) = A(t) \cdot R(t, \tau)
					\]
				\item[zu 2.]
					Differenzieren von \ref{Resolvente Eigenschaften} 3. nach $t_2$ gibt uns
					\[
						\frac{\partial R}{\partial t_2} E_n = \frac{\partial R}{\partial t_2} R(t_1, t_2) \cdot R(t_2, t_1)
					\] also \[
						0 = \left[ \frac{\partial R}{\partial t_2} R(t_1, t_2) \right] \cdot R(t_2, t_1) +
							R(t_1, t_2) \cdot \underbrace{\left[ \frac{\partial R}{\partial t_2}R(t_2, t_1) \right]}_{=A(t_2) \cdot R(t_2, t_1)} 
					\] woraus \[
						0 = \left [ \frac{\partial R}{\partial t_1}R(t_1, t_2) + R(t_1, t_2) \cdot A(t_2) \right] \cdot R(t_2, t_1)
					\] folgt.
					
					\ref{Resolvente Eigenschaften} 3. gibt uns, dass $R(t_1, t_2)$ 
					invertierbar ist ($R(t_2, t_1)$ ist die zugehörige Inverse). Also
					gilt:
					\[
						\frac{\partial R}{\partial t_2}R(t_1, t_2) = - R(t_1, t_2) \cdot A(t_2)
					\]
			\end{enumerate}
		\end{Beweis}
	\section{Gram'sche Kontrollmatrix}\label{Gram'sche Kontrollmatrix}
	
\newpage
\printbibliography
\newpage
\printindex
\end{document}

% ===REFERENZEN===
% Krieg - Gew. DGl
% Rannacher
