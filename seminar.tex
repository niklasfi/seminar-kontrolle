\documentclass[a4paper,twoside]{article}

\usepackage[utf8]{inputenc}
\usepackage[T1]{fontenc}
\usepackage{cmbright} %cmbright typeface
\usepackage{amsmath}
\usepackage{enumerate}
\usepackage{amssymb}
\usepackage{framed}
\usepackage[framed]{ntheorem}
\usepackage[normalem]{ulem}
\usepackage[pdfborder={1 1 0}]{hyperref}
\usepackage{fancyhdr}
\usepackage{todonotes}
\usepackage[ngerman]{babel}
\usepackage{makeidx}

\setlength{\headheight}{12.5pt}

\fancypagestyle{mystyle}{ %
	\fancyhf{} % remove everything
	\renewcommand{\headrulewidth}{0.25pt} % remove lines as well
	\renewcommand{\footrulewidth}{0.25pt}
	\fancyhead[LE,RO]{\thepage}
	\fancyhead[RE,LO]{\leftmark}
}
%TODO spacing sollte genauso wie von section sein.
\makeatletter
\newtheoremstyle{newBreak}%
  {\item[\rlap{\vbox{\hbox{\hskip\labelsep \theorem@headerfont
          \large ##1\ ##2\theorem@separator}\hbox{\strut}}}]}%
  {\item[\rlap{\vbox{\hbox{\hskip\labelsep \theorem@headerfont
          \large ##1\ ##2\ (##3)\theorem@separator}\hbox{\strut}}}]}
\makeatother

\theorembodyfont{}
\theoremstyle{newBreak}

\newcounter{thmc}[section] %theorem counter
\renewcommand{\thethmc}{\thesection.\arabic{thmc}}
\newframedtheorem{Definition}[thmc]{Definition}
\newtheorem{Bemerkung}[thmc]{Bemerkung}
\theoremsymbol{\rule{1ex}{1ex}}
\newframedtheorem{Satz}[thmc]{Satz}
\newframedtheorem{Lemma}[thmc]{Lemma}

\theoremstyle{nonumberplain}
\theoremsymbol{\rule{1ex}{1ex}}
\newtheorem{pro}{Beweis}

\newcommand{\N}[0]{\mathbb{N}}
\newcommand{\Z}[0]{\mathbb{Z}}
\newcommand{\Q}[0]{\mathbb{Q}}
\newcommand{\R}[0]{\mathbb{R}}
\newcommand{\C}[0]{\mathbb{C}}

\newcommand{\plainset}[1]{\left\{#1\right\}}
\newcommand{\ouptoset}[1]{\plainset{1, ..., #1}} %one up to set
\newcommand{\zuptoset}[1]{\plainset{0, ..., #1}} %zero up to set
\newcommand{\condset}[2]{\left\{ #1: \; #2\right\}} %condition set

\renewcommand{\Re}[0]{\operatorname{Re}}
\renewcommand{\Im}[0]{\operatorname{Im}}
\newcommand{\sgn}[0]{\operatorname{Sgn}}
\newcommand{\argmin}[0]{\operatorname{Argmin}}
\newcommand{\argmax}[0]{\operatorname{Argmax}}
\newcommand{\dist}[0]{\operatorname{Dist}}

\makeindex

\begin{document}
	\title{Seminararbeit zur Kontrolle}
	\author{Elisa Friebel, Marcel Goesmann, Niklas Fischer, betreut durch: Michael Herty}	
	\date{\today}
	\maketitle
	\begin{abstract}
		-- Zusammsenfassung --
	\end{abstract}
	\tableofcontents
	\newpage
	\pagestyle{mystyle}
\section{Definitionen}
	\begin{Bemerkung}
		In der ganzen Arbeit gelten die folgenden Konventionen:
		\begin{enumerate}
			\item $T_0$ und $T_1$ seien reelle Zahlen und es sei $T_0 < T_1$.
			\item $A: (T_0, T_1) \rightarrow \R^{n \times n}$ und $B: (T_0, T_1) 
				\rightarrow \R^{n \times m}$ seien stetige Funktionen.
			\item $x: (T_0, T_1) \rightarrow \R^n$ und $u: (T_0, T_1) 
				\rightarrow \R^m$ seien stetige Funktionen
		\end{enumerate}
	\end{Bemerkung}
	
	\begin{Definition}[Anfangswertproblem]\index{Anfangswertproblem}\index{AWP}
	Sei $b: (T_0, T_1) \rightarrow \R^n$ eine stetige Funktion. Eine Funktion 
	$x$ wird Lösung des \emph{Anfangswertproblems (auch: AWP)} \[
		\dot x(t) = A(t) x(t) + b(t)
	\] genannt, falls diese Gleichung für alle $t \in [T_0, T_1]$ erfüllt ist, und
	für gegebene $x_0 \in \R^n$ und $t_0 \in [T_0, T_1]$ die Gleichung $x(t_0) = 
	x_0$ gilt.
	
	Seien $b: (T_0, T_1) \rightarrow \R^n$ eine stetige Funktion, $x_0 \in \R^n$ 
	und $t_0 \in [T_0, T_1]$. Eine Funktion $x$ wird Lösung des 
	\emph{Anfangswertproblems (auch: AWP)}
	\[
		\dot x(t) = A(t)x(t) + b(t) \; \forall t \in [T_0, T_1] \quad \wedge \quad x(t_0) = x_0
	\]
	gennant, falls diese Gleichung erfüllt wird.
	
	Seien $b: (T_0, T_1) \rightarrow \R^n$ eine stetige Funktion, $x_0 \in \R^n$ und $t_0 \in [T_0, T_1]$.
	Ein System
	\begin{align}\label{gl:awp}
		\dot x(t) = A(t)x(t) + b(t) \; \forall t \in [T_0, T_1] \quad \wedge \quad x(t_0) = x_0
	\end{align} wird \emph{Anfangswertproblem (auch: AWP)} genannt. Eine Funktion $x$, die \eqref{gl:awp} erfüllt, wird \emph{Lösung} des Anfangswertproblems genannt.
	\end{Definition}
	
	\begin{Definition}[Lineares Kontrollsystem]\index{lineares Kontrollsystem}
		Ein System
		\begin{align*}%\label{gl:ks}
			\dot x(t) = A(t) x(t) + B(t) u(t),\quad t \in [T_0, T_1]
		\end{align*}
		wird \emph{lineares Kontrollsystem} genannt.
	\end{Definition}
	
	\begin{Definition}[kontrollierbar]\index{kontrollierbar}
		Seien $x_0, x_1 \in \R^n$. Ein lineares Kontrollsystem 
			\[
				\dot x(t) = A(t)x(t) + B(t)u(t)
			\]
		ist \emph{kontrollierbar}, falls es eine stetige Funktion $u$ gibt, sodass die Lösungfunktion $x$ des Anfangswertproblems 
		\begin{align*}
			\dot x(t) = A(t)x(t) + B(t)u(t) \; \forall t \in [T_0, T_1] \quad \wedge \quad x(T_0) = x_0
		\end{align*}
		in $T_1$ den Wert $x_1$ annimmt.
	\end{Definition}
	\begin{pro}
		TEST
	\end{pro}
	
\newpage
\printindex
\end{document}

% ===REFERENZEN===
% Krieg - Gew. DGl
% Rannacher
