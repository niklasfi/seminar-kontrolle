\documentclass[a4paper,twoside]{article}


\usepackage[utf8]{inputenc}
\usepackage[T1]{fontenc}
\usepackage{cmbright} %cmbright typeface
\usepackage{amsmath}
\usepackage{enumitem}
\usepackage{amssymb}
\usepackage{framed}
\usepackage[framed,thmmarks]{ntheorem}
\usepackage[normalem]{ulem}
\usepackage[pdfborder={1 1 0}]{hyperref}
\usepackage{fancyhdr}
\usepackage{todonotes}
\usepackage[ngerman]{babel}
\usepackage{makeidx}
\usepackage{biblatex}
\usepackage{graphicx}
\bibliography{seminar}
\setlength{\headheight}{12.5pt}

\fancypagestyle{mystyle}{ %
	\fancyhf{} % remove everything
	\renewcommand{\headrulewidth}{0.25pt} % remove lines as well
	\renewcommand{\footrulewidth}{0.25pt}
	\fancyhead[LE,RO]{\thepage}
	\fancyhead[RE,LO]{\leftmark}

}

	\addtolength{\headsep}{12pt}

%TODO spacing sollte genauso wie von section sein.
\makeatletter
\newtheoremstyle{newBreak}%
  {\item[\rlap{\vbox{\hbox{\hskip\labelsep \theorem@headerfont
          \large ##1\ ##2\theorem@separator}\hbox{\strut}}}]}%
  {\item[\rlap{\vbox{\hbox{\hskip\labelsep \theorem@headerfont
          \large ##1\ ##2\ ({##3})\theorem@separator}\hbox{\strut}}}]}
\makeatother
\theorembodyfont{}
\theoremstyle{newBreak}

\newcounter{thmc}[section] %theorem counter
\renewcommand{\thethmc}{\thesection.\arabic{thmc}}
\newframedtheorem{Definition}[thmc]{Definition}
\newtheorem{Bemerkung}[thmc]{Bemerkung}
\newtheorem{Beispiel}[thmc]{Beispiel}
\newframedtheorem{Exkurs}[thmc]{Exkurs}
\theoremsymbol{}
\newframedtheorem{Satz}[thmc]{Satz}
\newframedtheorem{Lemma}[thmc]{Lemma}
\newframedtheorem{Korollar}[thmc]{Korollar}

\theoremstyle{nonumberplain}

\theoremsymbol{\rule{1ex}{1ex}}
\newtheorem{Beweis}{Beweis}

\newcommand{\N}[0]{\mathbb{N}}
\newcommand{\Z}[0]{\mathbb{Z}}
\newcommand{\Q}[0]{\mathbb{Q}}
\newcommand{\R}[0]{\mathbb{R}}
\newcommand{\C}[0]{\mathbb{C}}
\newcommand{\G}[0]{\mathfrak{C}}
\newcommand{\E}[1][n]{\mathrm{E_{#1}}}
\newcommand{\Id}[1][n]{\mathrm{Id_{#1}}}

\newcommand{\plainset}[1]{\left\{#1\right\}}
\newcommand{\ouptoset}[1]{\plainset{1, ..., #1}} %one up to set
\newcommand{\zuptoset}[1]{\plainset{0, ..., #1}} %zero up to set
\newcommand{\condset}[2]{\left\{ #1: \; #2\right\}} %condition set

\renewcommand{\Re}[0]{\operatorname{Re}}
\renewcommand{\Im}[0]{\operatorname{Im}}
\newcommand{\sgn}[0]{\operatorname{Sgn}}
\newcommand{\argmin}[0]{\operatorname{Argmin}}
\newcommand{\argmax}[0]{\operatorname{Argmax}}
\newcommand{\dist}[0]{\operatorname{Dist}}
\newcommand{\rang}[0]{\operatorname{rang}}

\newenvironment{enumerateRef}{\begin{enumerate}[ref={\thethmc\alph*},label={(\alph*) }, leftmargin=*]}{\end{enumerate}}
\newenvironment{enumerateBew}{\mbox{}\begin{enumerate}[label={zu (\alph*):}, leftmargin=*]}{\end{enumerate}}
\makeindex

\begin{document}
	\title{Seminararbeit zur Kontrolle}
	\author{Elisa Friebel, Marcel Goesmann, Niklas Fischer}	
	\date{\today}
	\maketitle
	\begin{abstract}
		Diese von Michel Herty betreute Arbeit ist eine Ausarbeitung des Kapitels "`Finite-dimensional linear control 
		systems"' aus dem Buch "`Control and Nonlinearity"' von Jean-Michel Coron \cite{Coron2007}.
	\end{abstract}
	\tableofcontents
	\newpage
	\pagestyle{mystyle}


\setcounter{section}{-1}

\section{Einleitung}

\begin{center}
\begin{figure}[ht]
\includegraphics[clip,scale=0.7]{umlauf}\caption{Trajektorie der Dawn Sonde \cite{NASA2009}}
\end{figure}
\end{center}

In unserem Seminar wollen wir eine Einführung in die lineare Kontrolltheorie geben. Sie befasst sich hauptsächlich mit der Steuerung von Differentialgleichungen. Das heißt, die Trajektorie einer Lösung einer Differentialgleichung soll zu einer bestimmten Zeit $t_1$ einen bestimmten Zustand $x_1$ erreichen. Anwendung findet die Theorie zum Beispiel bei der Steuerung von Raumfahrzeugen, wie der Dawn Sonde. Ziel der Mission war es die Planeten Vesta und Ceres zu besuchen. Es war also eine Trajektorie gesucht, die von der Erde ($x_0$) im Zeitpunkt $t_0$ ausgehend Vesta ($x_1$) erreicht. 

\newpage

\section{Definitionen}
	\begin{Bemerkung}
		In der ganzen Arbeit gelten die folgenden Konventionen:
		\begin{enumerate}
			\item $T_0$ und $T_1$ seien reelle Zahlen und es sei $T_0 < T_1$.
			\item $A: [T_0, T_1] \rightarrow \R^{n \times n}$ und $B: [T_0, T_1] 
				\rightarrow \R^{n \times m}$ seien stetige Funktionen.
			\item $x: [T_0, T_1] \rightarrow \R^n$ ist eine stetige Funktion
		\end{enumerate}
	\end{Bemerkung}
	Betrachte zunächst folgende altbekannte Definition:
	\begin{Definition}[Anfangswertproblem]\label{AWP}\index{Anfangswertproblem}
	Sei $b: [T_0, T_1] \rightarrow \R^n$ eine stetige Funktion. Eine Funktion 
	$x$ wird Lösung des \emph{Anfangswertproblems (auch: AWP)} \[
		\dot x(t) = A(t) x(t) + b(t)
	\] genannt, falls diese Gleichung für alle $t \in [T_0, T_1]$ erfüllt ist, und
	für gegebene $x_0 \in \R^n$ und $t_0 \in [T_0, T_1]$ die Gleichung $x(t_0) = 
	x_0$ gilt.
	\end{Definition}
	Nun können lineare Kontrollsysteme beziehungsweise die Kontrollierbarkeit definiert werden, welche einen zentralen Punkt in der Arbeit einnehmen. 
	
	\begin{Definition}[lineares Kontrollsystem, kontrollierbar]\index{lineares Kontrollsystem}\index{kontrollierbar}
		Ein lineares Kontrollsystem 
			\[
				\dot x(t) = A(t)x(t) + B(t)u(t),\quad t \in [T_0, T_1]
			\]
		ist \emph{kontrollierbar}, falls es für alle $x_0, x_1 \in \R^n$ eine stetige Funktion $u$ gibt, sodass die Lösungsfunktion $x$ des Anfangswertproblems 
		\begin{align*}
			\dot x(t) = A(t)x(t) + B(t)u(t) \; \forall t \in [T_0, T_1] \quad \text{mit} \quad x(T_0) = x_0
		\end{align*}
		in $T_1$ den Wert $x_1$ annimmt.
	\end{Definition}
	Anschaulich kann man die Kontrollierbarkeit linearer Kontrollsysteme so interpretieren, dass zum 
	Beispiel ein Raumfahrzeug mit einer Steuerung $u$ vom Startpunkt $x_0$ zu 
	einem Punkt beliebigen anderen Punkt $x_1$ im Raum gebracht werden können muss. 
	Dabei modelliert $A$ die 
	"`natürlichen"' Kräfte auf das Objekt, und $B$ bildet Eigenschaften der 
	Steuerungseinheit des Raumschiffes, beispielsweise die Anordnung der 
	Triebwerke, ab. $u$ wiederum stellt die konkrete Steuerung dar, also zu 
	welchem Zeitpunkt die Triebwerke wie angesteuert werden sollen. 
	In der Raumfahrt geht es darum eine Raumsonde von der Erde aus zu einem einzigen bestimmten Punkt zu steuern.
	Im Gegensatz dazu ist es für die Kontrollierbarkeit eines Systems wichtig, 
	dass die Sonde von jedem Punkt ausgehend jeden anderen Erreichen kann.
	
	
	\section{Die Resolvente}
	\subsection*{Einleitung}
		Von besonderem Interesse ist jetzt natürlich ein Kriterium für die 
		Kontrolleribarkeit eines linearen Kontrollsystems. Dieses wird in
		Abschnitt \ref{Gram'sche Kontrollmatrix} geliefert. Hierfür ist allerdings
		etwas Vorarbeit notwendig.
		
		Für lineare Differentialgleichungssysteme \[
			\dot x(t) = A \cdot x(t)
		\] mit konstanten $A \in \R ^ {n \times n}$ ist es mithilfe der 
		Exponentialfunktion einfach ein Fundamentalsystem zu bestimmen, doch was
		geschieht, wenn $A$ wie in der in \ref{AWP} gegebenen Definition eines 
		Anfangswertproblems abhängig von $t$ ist? In diesem Fall ist die
		Fundamentalmatrix im allgemeinen nicht elementar bestimmbar.
		
		Diese Klippe wird mit Resolvente $R$, die uns eine eindeutig bestimmte 
		Fundamentalmatrix eines Anfangswertproblems	liefert, geschickt umschifft.
		Mit der Resolvente und dem zentralen Satz aus Abschnitt
		\ref{Gram'sche Kontrollmatrix} können kann dann überpüft werden, ob das 
		lineare Kontrollproblem kontrollierbar ist. 
	\subsection*{Definition}
		Sei $\Phi:\R \rightarrow \R^{n \times n}$ die Funktion, die das homogene Anfangswertproblem
		\[
			\dot M(t) = A(t)\cdot M(t) \; \forall t \in [T_0, T_1]  \quad \text{mit} \quad M(\tau) = \E
		\]
		löst. Hierbei steht $\E \in \R^{n \times n}$ für die Einheitsmatrix. Offensichtlich ist $\Phi$ von der Wahl von $\tau$ abhängig. Definiere
		$\tilde R$ als die Funktion, die $\tau$ auf die Funktion $\Phi$ 
		abbildet:
		\[
			\tilde R : [T_0, T_1] \rightarrow C^0( [T_0, T_1],\R^{n \times n}), 
			\tau \mapsto \Phi
		\]
		Jetzt kann die Resolvente einfach definiert werden:
		
		\begin{Definition}[Resolvente]\index{Resolvente}
			Die Funktion 
			\[
				R: [T_0, T_1] \times [T_0, T_1] \rightarrow \R^{n \times n},
				(t, \tau) \mapsto \tilde R(\tau)(t)
			\]
			heißt Resolvente.
		\end{Definition}

		\begin{Beispiel}
			Gegeben sei die Differentialgleichung 
			\begin{align*}
				\dot M(t) = A(t) \cdot M(T) \quad \text{mit} \quad A(t) :=
				\begin{pmatrix}
					t & -1 \\
					1 & t
				\end{pmatrix}
				\label{Resolvente:Beispiel:star}\tag{$\star$}
			\end{align*} von der wir die Resolvente bestimmen wollen.
			Die Differentialgleichung \eqref{Resolvente:Beispiel:star} kann mit
			\begin{align*}
				& z(t) = x_1(t) + i x_2(t)
				\label{Resolvente:Beispiel:star2}\tag{$\star\star$} \\
				\Rightarrow & \dot z(t) = \dot x_1(t) + i \dot x_2(t)
			\end{align*} umgeschrieben werden:
			\begin{align*}
				\dot z(t) =  & ( t x_1(t) -x_2(t))+ i ( x_1(t) +t x_2(t)) \\
				= & t ( x_1(t) + i (x_2) ) + i (x_1(t) + i x_2(t)) \\
				= & t \cdot z(t) + i \cdot z(t)\\
			 	= & (t + i) \cdot z(t)
			\end{align*}
			Betrachte also das Anfangswertproblem
			\begin{align*}
				\dot z(t) = (t + i)  \cdot z(t) \quad \text{mit} \quad z(t_2) = 1
			\end{align*}
			welches mit dem Korollar über lineare Differentialgleichungen aus \cite{KriegWalcher2010}[S. 14]
			\[
				z(t) = z(t_2) \exp \left( \frac{t^2}2 - \frac{t_2^2}2 + i t - i t_2 \right)
			\]
			ergibt. Mit \eqref{Resolvente:Beispiel:star2} folgt auch \[
				x_1(t) = \Re( z(t)) \quad \text{und} \quad x_2(t) = \Im(z(t)).
			\]
			Dann ist die Resolvente gegeben durch 
			\[
				R(t, t_2) =
				\begin{pmatrix}
					\cos(t-t_2) \exp \left( \frac{t^2}2 - \frac{t_2^2}2 \right) &
					-\sin(t-t_2) \exp \left( \frac{t^2}2 - \frac{t_2^2}2 \right) \\
					\sin(t-t_2) \exp \left( \frac{t^2}2 - \frac{t_2^2}2 \right) &
					\phantom{-}\cos(t-t_2) \exp \left( \frac{t^2}2 - \frac{t_2^2}2 \right)
				\end{pmatrix}
			\]
			
		\end{Beispiel}
	\subsection*{Eigenschaften}
		\begin{Lemma}
			$R$ ist stetig.
		\end{Lemma}
		\begin{Beweis}
			Die Stetigkeit von $R$ in der 1. Komponente ($t$) folgt aus der Stetigkeit
			jeder Lösungsfunktion $\Phi$ des von $\tau$ abhängigen Anfangswertproblems.
		
			Die Stetigkeit in der 2. Komponente ($\tau$) folgt aus dem Satz über die
			Stetige Abhängigkeit aus \cite{KriegWalcher2010}.
			Für $\tau_1, \tau_2 \in [T_0,T_1]$ betrachte die Anfangswertprobleme
			\[
				\dot M(t) = A(t)\cdot M(t) \; \forall t \in [T_0, T_1]  \quad \text{mit} \quad M(\tau_1) = \E
			\]
			und
			\[
				\dot M(t) = A(t)\cdot M(t) \; \forall t \in [T_0, T_1]  \quad \text{mit} \quad M(\tau_2) = \E
			\]
			mit Lösungen $\Phi_1$ bzw. $\Phi_2$.
			Dann existiert zu jedem $\varepsilon > 0 $ ein $\delta := | \tau_1 - \tau_2 |$
			sodass \[
				|| \Phi_1(t) - \Phi_2(t) || < \varepsilon \; \forall t \in [T_0, T_1]
			\]
			dies ist äquivalent zur Stetigkeit von $R$ in der 2. Komponente.
			Da $R$ in beiden seiner Komponenten stetig ist, mit $R$ stetig.
		\end{Beweis}
		
		\begin{Lemma}\label{Resolvente Eigenschaften}
			Die Resolvente erfüllt die folgenden Eigenschaften
			\begin{enumerateRef}
				\item\label{Resolvente Eigenschaften:1} $	R(t_1,t_1) = \E$
				\item\label{Resolvente Eigenschaften:2} $ R(t_1, t_2) \cdot R(t_2, t_3) = R(t_1, t_3)$
				\item\label{Resolvente Eigenschaften:3} $ R(t_1, t_2) \cdot R(t_2, t_1) = \E$
			\end{enumerateRef}
			für alle $t_1, t_2, t_3 \in [T_0, T_1]$.
		\end{Lemma}
		
		\begin{Beweis}
		\begin{enumerateBew}
			\item Betrachten wir das zu $R(t_1, t_1) $ gehörige 
			Anfangswertproblem \[
				\dot M(t) = A(t) \cdot M(t) \; \forall t \in [T_0, T_1] \quad \text{mit} \quad M(t_1) = \E
			\] mit Lösung $\Phi$.
			Nun gilt: \[
				R(t_1, t_1) = \tilde R (t_1)(t_1) = \Phi(t_1) = \E.
			\]
			\item 
				Betrachte die wie in \cite{KriegWalcher2010}[S. 43] beschriebene Vektorraumeigenschaft von homogenen Differentialgleichungen. Interessant ist, dass die Abbildung die einem Anfangswert eine Lösungsfunktion zuordnet ein Isomorphismus ist. Wenn also die Funktion $\Phi$ das Anfangswertproblem \[
					\dot M(t) = A(t) M(t) \quad \text{mit} \quad M(t_0) = C
				\] löst,
				so löst $\Phi \cdot D$ das Anfangswertproblem \[
					\dot M(t) = A(t) M(t) \quad \text{mit} \quad M(t_0) = C \cdot D.
				\]

				Mit dieser Eigenschaft kann die Aussage gezeigt werden. Betrachte $\tilde R(t_2)$ und $\tilde R(t_3)$. Diese beiden Funktionen sind die Lösungen der Anfangswertprobleme 
				\[
					\dot M_2(t) = A(t) M_2(t) \quad \text{mit} \quad M_2(t_2) = \E.
				\]				
				beziehungsweise
				\[
					\dot M_3(t) = A(t) M_3(t) \quad \text{mit} \quad M_3(t_3) = \E
				\]
				Also löst $\tilde R(t_2) \cdot R(t_2,t_3)$ das Anfangswertproblem
				\[
					\dot M_2(t) = A(t)M_2(t) \quad \text{mit} \quad M_2(t_2) = R(t_2,t_3).
				\]
				Wertet man nun $\tilde R(t_2) \cdot R(t_2,t_3)$ in $t_2$ aus, so erhält man
				\[
					R(t_2,t_2) \cdot R(t_2,t_3) = R(t_2,t_3) = \tilde R(t_3)(t_2).
				\]
				Da die Funktion $(t,x) \mapsto A(t) x$ in $x$ lokal einer Lipschitzbedingung genügt, sind die Lösungen von Anfangswertproblemen eindeutig bestimmt. Aus der obigen Gleichung folgt also, dass die Bahnen von $\tilde R(t_3)$ und $\tilde R(t_2) \cdot R(t_2,t_3)$ in $t_2$ übereinstimmen. Daraus folgt, dass beide Bahnen identisch sein müssen, was dazu führt, dass die Lösungsfunktionen in $t_1$ ausgewertet auch übereinstimmen müssen.
			\item 
				\[
					R(t_1, t_2) \cdot R(t_2, t_1) \overset{\ref{Resolvente Eigenschaften:2}}= R(t_1, t_1) \overset{\ref{Resolvente Eigenschaften:1}}= E_n
				\]
		\end{enumerateBew}
		\end{Beweis}
		
		\begin{Lemma}\label{Resolvente Ableitung}
			Für die Ableitungen der Resolvente gilt:
			\begin{enumerateRef}
				\item\label{Resolvente Ableitung:1} $ \frac{\partial R}{\partial t} (t, \tau) = A(t) \cdot R(t,\tau) \; \forall t,\tau \in [T_0, T_1]$
				\item\label{Resolvente Ableitung:2} $ \frac{\partial R}{\partial \tau} (t, \tau) = - R(t, \tau) \cdot A(\tau) \; \forall t,\tau \in [T_0, T_1]$
			\end{enumerateRef}
		\end{Lemma}
		
		\begin{Beweis}
			\begin{enumerateBew}
				\item
					Die Definition der Resolvente mit ihrem Anfangswertproblem liefert uns
					die Interpretation \[
						R(t, \tau ) = E_n + \int_\tau^t A(s) \cdot R(s, \tau) ds.
					\]
					Differenzieren liefert
					\[
						\frac{\partial R}{\partial t}R(t,\tau) = A(t) \cdot R(t, \tau).
					\]
				\item
					Differenzieren von \ref{Resolvente Eigenschaften:3} nach $t_2$ gibt uns
					\[
						\frac{\partial R}{\partial t_2} E_n = \frac{\partial R}{\partial t_2} R(t_1, t_2) \cdot R(t_2, t_1)
					\] also \[
						0 = \left[ \frac{\partial R}{\partial t_2} R(t_1, t_2) \right] \cdot R(t_2, t_1) +
							R(t_1, t_2) \cdot \underbrace{\left[ \frac{\partial R}{\partial t_2}R(t_2, t_1) \right]}_{=A(t_2) \cdot R(t_2, t_1)} 
					\] woraus \[
						0 = \left [ \frac{\partial R}{\partial t_1}R(t_1, t_2) + R(t_1, t_2) \cdot A(t_2) \right] \cdot R(t_2, t_1)
					\] folgt.
					
					\ref{Resolvente Eigenschaften:3} liefert, dass $R(t_2, t_1)$ 
					invertierbar ist ($R(t_1, t_2)$ ist die zugehörige Inverse). Also
					gilt:
					\[
						\frac{\partial R}{\partial t_2}R(t_1, t_2) = - R(t_1, t_2) \cdot A(t_2),
					\]
					was zu zeigen war.
			\end{enumerateBew}
		\end{Beweis}

		\begin{Lemma}\label{Resolvente AWP}
			Für die Lösung $x$ des Anfangswertproblems \begin{align*}
				\dot x(t) = A(t) x(t) + b(t) \; \forall t \in [T_0, T_1] \quad \text{mit} \quad x(T_0) = x^0
				\label{Resolvente AWP:star}\tag{$\star$}
			\end{align*} gilt 
			\begin{enumerateRef}
				\item\label{Resolvente AWP:1} $x(t_1) = R(t_1, t_0) x(t_0) + \int_{t_0}^{t_1} R(t_1,\tau) b(\tau) d\tau$
				\item\label{Resolvente AWP:2} $x(t) = R(t, T_0) x^0+ \int_{T_0}^t R(t, \tau)  b(\tau)d\tau$.
			\end{enumerateRef} für alle $t_0, t_1, t \in [T_0, T_1]$
		\end{Lemma}
		\begin{Beweis}
			\begin{enumerateBew}
				\item
					 Differenzieren von \ref{Resolvente AWP:1} nach $t_1$ ergibt 								\begin{align*}
						& \frac \partial{\partial t_1} x(t_1) \\
						 = & \frac \partial{\partial t_1} \left(R(t_1, t_0)x(t_0) + \int_{t_0}^{t_1} R(t_1,\tau) b(\tau) d\tau \right)\\
						 = & \frac \partial {\partial t_1}  R(t_1,t_0)x(t_0) + \frac \partial {\partial t_1} \int_{t_0}^{t_1} R(t_1,\tau) b(\tau) d\tau.
						%LEIBNITZ REGEL FÜR PARAMETERINTEGRALE
						\intertext{Mit der Leibnitzregel für Parameterintegrale folgt}
						= & A(t_1)R(t_1,t_0)x(t_0) + \\ & \qquad \int_{t_0}^{t_1} \frac \partial {\partial t_1} R(t_1,\tau)b(\tau)d\tau +  R(t_1,t_1)b(t_0) \cdot 1 - R(t_1,t_0)b(t_0) \cdot 0 \\
						= & A(t_1)R(t_1,t_0)x(t_0) + \int_{t_0}^{t_1} A(t_1) R(t_1,\tau)b(\tau)d\tau +  R(t_1,t_1)b(t_1) \\
						= & A(t_1)\left[R(t_1,t_0)x(t_0) + \int_{t_0}^{t_1} R(t_1,\tau)b(\tau )d\tau \right] + b(t_1) \\
						= & A(t_1) x(t_1) + b(t_1).
					\end{align*}
					Also erfüllt \ref{Resolvente AWP:1} das Anfangswertproblem \eqref{Resolvente AWP:star}.
				\item
					Spezialfall von \ref{Resolvente AWP:1} mit $t_0:= T_0$ und $t_1 := t$ 
			\end{enumerateBew}
		\end{Beweis}
		
	\section{Gram'sche Kontrollmatrix}\label{Gram'sche Kontrollmatrix}
		\begin{Definition}[Gram'sche Kontrollmatrix]\index{Gram'sche Kontrollmatrix}
			Für ein Kontrollsystem \[
				\dot x(t) = A(t) x(t) + B(t) u(t)
			\]
			heißt die Matrix \[
				\G:=\int_{T_0}^{T_1} R(T_1, \tau) \cdot B(\tau) \cdot B(\tau)^T\cdot R(T_1, \tau)^T d\tau \in \R^{n \times n}
			\] Gram'sche Kontrollmatrix
		\end{Definition}
		
		\begin{Lemma}\label{Gram'sche Kontrollmatrix:definitheit}
			\begin{enumerateRef}
				\item $\G$ ist symmetrisch\label{Gram'sche Kontrollmatrix:Eigenschaften:symmetrie}
				\item $\G$ ist positiv semidefinit\label{Gram'sche Kontrollmatrix:Eigenschaften:definitheit}
			\end{enumerateRef}
		\end{Lemma}
		
		\begin{Beweis}
			\begin{enumerateBew}
			\item Für eine beliebige quadratische Matrix $M = (m)_{i,j}\in \R^{n \times n}$ ist $M\cdot M^T$
			symmetrisch. Da das Integral einer Matrix als Integral über die Matrixeinträge 
			definiert ist, und $m_{i,j} = m_{j,i}$ gilt, ist das Integral über eine symmetrische
			Matrix wieder symmetrisch. Wähle nun $M := R(T_1,\tau)\cdot B(\tau)$, dann ist
			$\G$ symmetrisch.
			\item Es gilt für $x \in \R^n$:
				\begin{align*}
				x^T \G x = & x^T \int_{T_0}^{T_1} R(T_1, \tau) B(\tau) B(\tau)^T R(T_1, \tau)^T d\tau \; x\\
					=& \int_{T_0}^{T_1} \langle B(\tau)^T R(T_1, \tau)^T x, B(\tau)^T R(T_1, \tau)^T x\rangle d\tau \\
					=& \int_{T_0}^{T_1} | B(\tau)^T R(T_1, \tau)^T x|^2 d\tau \geq 0.
			\end{align*}
			Damit ist $\G$ positiv semidefinit.
		\end{enumerateBew}
		\end{Beweis}
		
		\begin{Satz}\label{Kontrollierbarkeit}
			Ein lineares Kontrollsystem \[
				\dot x(t) = A(t) \cdot x(t) + B(t) \cdot u(t)
			\] ist genau dann kontrollierbar, wenn seine
			Gram'sche Kontrollmatrix regulär ist.
		\end{Satz}
		
		\begin{Beweis}
			\begin{enumerate}
				\item["`$\Leftarrow$"': ]
					Sei $\G$ invertierbar, $x^0, x^1 \in \R^n$ beiliebig. Definiere \[
						\bar u: (T_0, T_1) \rightarrow \R^m, \tau \mapsto B(\tau)^T\cdot R(T_1,\tau)^T\G^{-1}\cdot (x^1-R(T_1, T_0)\cdot x^0)
					\]
					Dann ist $\bar u$ stetig, da $\G^{-1}(x^1-R(T_1, T_0)\cdot x^0)$ eine
					von $\tau$ unabhängige Konstante ist, und $B$, sowie $R$ stetig sind.
					
					Sei $\bar x$ die Lösung des Anfangswertproblems \[
						\dot {\bar x}(t) = A(t) \cdot \bar x(t) + B(t) \cdot \bar u (t) \quad \text{mit} \quad \bar x (T_0) = x^0
					\]
					
					Nach \ref{Resolvente AWP:2} gilt: \begin{align*}
						\bar x(T_1) = & R(T_1, T_0) x^0 + \int_{T_0}^{T_1} R(T_1, \tau )  B(\tau) \bar u(\tau) d\tau \\
						 = & R(T_1, T_0) x^0 + \\
						 & \qquad \int_{T_0}^{T_1} R(T_1, \tau ) B(\tau) B(\tau)^T R(T_1,\tau)^T\G^{-1}(x^1-R(T_1, T_0)x^0) d\tau\\
						 = & R(T_1, T_0) x^0 + \\
						 & \qquad \underbrace{\int_{T_0}^{T_1} R(T_1, \tau ) B(\tau) B(\tau)^T R(T_1,\tau)^T d\tau}_{=\G} \; \G^{-1}(x^1-R(T_1, T_0)x^0) \\
						 = & R(T_1, T_0) x^0 + x^1-R(T_1, T_0)x^0 \\
						 = & x^1
					\end{align*}
					
					also gibt es eine Steuerung $\bar u$, die die gewünschten Bedingungen erfüllt.
					
				\item["`$\Rightarrow$"': ]
					Falls $\G$ singulär ist, gibt es ein $y \in \R^n\setminus \plainset{0}$ mit \[
						\G y = 0.
					\]
					Also gilt auch
					\begin{align*}
						0 = y^T \G y = &\int_{T_0}^{T_1} y^T R(T_1,y)B(\tau)B(\tau)^T R(T_1, \tau)^T y d\tau \\
						= & \int_{T_0}^{T_1} | B(\tau)^T R(T_1, \tau)^T y |^2 d\tau.
					\end{align*}
					Daraus folgt 
					\begin{align*}
						y^T R(T_1, \tau) B(\tau) = 0 \tag{$\star$}\label{eqn:Gramsche Kontrollmatrix:star}
					\end{align*}
					da $\tau \mapsto B(\tau)^T R(T_1, \tau)^T y$ stetig ist.
					Sei nun $u: [T_0, T_1] \rightarrow R^m$ eine beliebige stetige Funktion. Dazu sei $x$ die Lösung des Anfangswertproblems \[
						\dot x(t) = A(t) x(t) + B(t) u(t) \quad \text{mit} \quad x(T_1) = 0 = x^1.
					\]
					Dann gilt wieder mit \ref{Resolvente AWP:2} \begin{align*}
						x(t) = & R(t, T_0) \cdot x^0+ \int_{T_0}^t R(t, \tau) \cdot B(\tau) u(\tau) d\tau \\
								 \overset{x^0=0}= & \int_{T_0}^t R(t, \tau) \cdot B(\tau) u(\tau) d\tau.
								 \tag{$\star\star$}\label{eqn:Gramsche Kontrollmatrix:star2}
					\end{align*}
					
					Aus \eqref{eqn:Gramsche Kontrollmatrix:star} und \eqref{eqn:Gramsche Kontrollmatrix:star2} folgt
					\begin{align*}
						y^T x(t) = & y^T \int_{T_0}^t R(t, \tau) \cdot B(\tau) u(\tau) d\tau \\
						= & \int_{T_0}^t \underbrace{ y^T R(t, \tau) \cdot B(\tau)}_{=0} u(\tau) d\tau = 0
					\end{align*}
					also, dass $ y^T x(T_1) = 0$ für beliebige $u$ gilt.
					
					Es existiert aber ein $x^1 \in \R^n$ mit $y^T x^1 \neq 0$, beispielsweise $x^1 = y$.
					
					Wenn das Kontrollproblem kontrollierbar wäre würde es eine Steuerung $u$ geben, sodass $x(T_1) = x^1$.
					Daraus würde dann aber auch \[0 = y^T x(T_1) = y^T x^1 \neq 0\] folgen, was ein Widerspruch ist.
					
					Also kann das Kontrollproblem nicht kontrollierbar sein, wenn $\G$ singulär ist.
			\end{enumerate}
		\end{Beweis}
		Mit $\bar u$ ist jetzt also eine allgemeine Steuerung gefunden. Aber $\bar u$ 
		erfüllt auch noch eine Minimalitätsbedingung:
		
		\begin{Satz}\label{Gramsche Minimalitaet}
			Seien $T_0, T_1\in \R$ sowie $x^0, x^1 \in \R^n$. Für ein beliebiges Kontrollproblem
			\[
				\dot x = A(t) x(t) + B(t) u(t) \; \forall t\in [T_0, T_1] \quad \text{mit} \quad x(T_0)=x^0, \; x(T_1)=x^1
			\]
			mit Lösung $u$ gilt
			\begin{enumerateRef}
				\item $\int_{T_0}^{T_1} |\bar u(\tau)|^2 d\tau < \int_{T_0}^{T_1} |u(\tau)|^2d\tau$, falls $u \neq \bar u$ und
				\item $\int_{T_0}^{T_1} |\bar u(\tau)|^2 d\tau = \int_{T_0}^{T_1} |u(\tau)|^2d\tau$, falls $u = \bar u$
			\end{enumerateRef}
			wobei
			\[
				\bar u: (T_0, T_1) \rightarrow R^m, \tau \mapsto B(\tau)^T\cdot R(T_1,\tau)^T\G^{-1}\cdot (x^1-R(T_1, T_0)\cdot x^0)
			\]
		\end{Satz}
		
		\begin{Beweis}
			Definiere \[
				v := u - \bar u.
			\]
			Betrachte zunächst $|u(t)|^2$
			\begin{align*}
				|u(t)|^2 = & |\bar u(t) + v(t)|^2 \\
					= & \langle \bar u(t) + v(t), \bar u(t) + v(t)\rangle \\
					= & \langle \bar u(t), \bar u(t)\rangle + 2 \langle \bar u(t), v(t)\rangle + \langle v(t),v(t) \rangle \\
					= & |\bar u(t) |^2 + 2 \bar u(t)^T v(t) + |v(t)|^2
			\end{align*}
			Also gilt auch
			\[
				\int_{T_0}^{T_1}|u(\tau)d\tau|^2 = \int_{T_0}^{T_1}|\bar u(\tau) |^2 + 2 \int_{T_0}^{T_1}\bar u(\tau)^T v(\tau) d\tau+ \int_{T_0}^{T_1}|v(\tau)|^2 d\tau.
			\]
		Für $\int_{T_0}^{T_1}\bar u(\tau)^T v(\tau) d\tau$ aber gilt
		\begin{align*}
			& \int_{T_0}^{T_1}\bar u(\tau)^T v(\tau) d\tau \\
			\overset{\text{Def. }u}= & \int_{T_0}^{T_1}(B(\tau)^T R(T_1,\tau)^T\G^{-1} (x^1-R(T_1, T_0) x^0))^T v(\tau) d\tau \\
			\overset{\G^{-1} \text{ sym.}}= & \int_{T_0}^{T_1} (x^1-R(T_1,T_0)x^0)^T \G^{-1}R(T_1,\tau)B(\tau)v(\tau)d\tau \\
			= \;\;\,& (x^1-R(T_1,T_0)x^0)^T \G^{-1} \int_{T_0}^{T_1} R(T_1,\tau)B(\tau)v(\tau)d\tau
		\end{align*}
		und für $\int_{T_0}^{T_1} R(T_1,\tau)B(\tau)v(\tau)d\tau$ gilt 
		\begin{align*}
			 & \int_{T_0}^{T_1} R(T_1,\tau)B(\tau)v(\tau)d\tau \\
			=& \int_{T_0}^{T_1} R(T_1,\tau)B(\tau)u(\tau)d\tau - \int_{T_0}^{T_1} R(T_1,\tau)B(\tau)\bar u(\tau)d\tau \\
			\overset{\eqref{eqn:Gramsche Minimalitaet:star}}=& (x(T_1) - R(T_1,T_0)x(T_0)) - (\bar x(T_1)-R(T_1,T_0) \bar x(T_0)) \\
			\overset{x, \bar x \text{ Lsg.}}=& (x^1-R(T_1,T_0)x^0) - (x^1-R(T_1,T_0)x^0) \\
			=& 0
			\tag{$\star\star$}\label{eqn:Gramsche Minimalitaet:star2} 
		\end{align*}
		wobei \eqref{eqn:Gramsche Minimalitaet:star} aus Umstellen von \ref{Resolvente AWP:2} folgt:
		\begin{align*}
			  & x(T_1) =R(T_1,T_0) x^0 + \int^{T_1}_{T_0}R(T_1,\tau)B(\tau)u(\tau)d\tau \\
			\Leftrightarrow {}&  x(T_1) - R(T_1,T_0) x^0 = \int^{T_1}_{T_0}R(T_1,\tau)B(\tau)u(\tau)d\tau
			\tag{$\star$}\label{eqn:Gramsche Minimalitaet:star} .
		\end{align*}
		Mit \eqref{eqn:Gramsche Minimalitaet:star2} gilt also
		\[
			\int_{T_0}^{T_1} R(T_1,\tau)B(\tau)v(\tau)d\tau = 0
		\]
		und damit
		\[
			\int_{T_0}^{T_1}|u(\tau)d\tau|^2 = \int_{T_0}^{T_1}|\bar u(\tau) |^2 +  \underbrace{\int_{T_0}^{T_1}|v(\tau)|^2 d\tau.}_{\geq 0}
		\]
		\end{Beweis}
		$\bar u$ ist also diejenige Steuerung, sodass $\int_{T_0}^{T_1} |u(\tau)|^2d\tau$ minimal ist.
		Anschaulich ist der Vorteil dieser Bedingung auch sofort klar. Wenn zum 
		Beispiel wie anfangs ein Raumfahrzeug gesteuert werden sollte, so sorgt diese
		Anforderung dafür, dass möglichst wenig Energie für die Steuerung verbraucht
		wird.
		
\section{Kalman Rang Bedingung}

Im Allgemeinen ist es schwer  die Gram'sche Matrix auszurechnen. Die Resolvente ist meist nicht einfach zu berechnen.

Also suchen wir ein handlicheres Kriterium. Dafür ziehen wir uns zunächst auf den zeitunabhängigen Fall zurück. Seien also $A$ und $B$ unabhängig von $t$. Es ist nicht klar, ob dann die Kontrollierbarkeit auch nicht von der Zeit abhängt. Schließlich gehen Anfangs- und Endpunkt in die Lösung mit ein.

Wir betrachten nun das lineare, zeitunabhänige System $\dot{x}=Ax+Bu$ in $[T_0,T_1]$. Ein leicht zu kontrollierbareres Kriterium ist gegeben durch das folgende

\begin{Satz}[Kalman Rang Bedingung]\label{Kalman Rang Bedingung}
  Das lineare, zeitunabhängige System $\dot{x}=Ax+Bu$ ist in $[T_0,T_1]$ kontrollierbar genau dann wenn
\[
  \rang(A^0 \mid A^1B \mid A^2B \mid \dots  \mid A^{n-1}B)=n
\]
\end{Satz}

\begin{Bemerkung}
  Dieser Satz zeigt, dass die Kontrolle des System nicht vom gewählten Intervall abhängt, also komplett zeitunabhängig ist. Wenn ein System auf $[T_0,T_1]$ kontrollierbar ist, dann auch auf $[\tilde{T_0},\tilde{T_1}]$.
\end{Bemerkung}

\begin{Beweis}[zu \ref{Kalman Rang Bedingung}]
 \begin{enumerate}
  \item["`$\Leftarrow$"']
    Da $A$ zeitunabhängig ist können wir die DGL leicht lösen und erhalten die Resolvente $R(t_1,t_2)=e^{(t_1-t_2)A} \; \forall  (t_1,t_2) \in [T_0,T_1]^2$ oder $R(t_1,t_2)=e^{(t_1-t_2)A} \; \forall \: (t_1,t_2) \in [T_0,T_1]^2$

    Bestimmen der Gramschen Matrix
    \begin{align*}
      \G=&\int\limits_{T_0}^{T_1} R(T_1,\tau \cdot B(\tau) \cdot B(\tau)^T \cdot R(T_1,\tau)^T d\tau \\
	=&\int\limits_{T_0}^{T_1} e^{(t_1-\tau)A} B B^T (e^{(t_1-\tau)A})^T d\tau \\
	=&\int\limits_{T_0}^{T_1} e^{(t_1-\tau)A} B B^T e^{(t_1-\tau)A^T} d\tau
    \end{align*}

		Bewiesen wird die Kontraktion, also wird angenommen: $\dot x = Ax+Bu$ ist nicht kontrollierbar in $[T_0,T_1]$. Dann ist $\G$ nicht invertierbar und hat somit einen nicht trivialen Kern. Es exisitert also $y \in \R^n \setminus \{0\}$ mit $\G y=0$
    Also auch
     \[
	y^T \G y = 0 
     \]
     und somit
    \[ 
      0 = c^T \G y = \int\limits_{T_0}^{T_1} \underbrace{|B^T e^{(T_1-\tau)A^T} y |^2}_{ \geq 0} d\tau.
    \]
      Aus der Stetigkeit folgt dann 
     \[  
      B^T e^{(T_1-\tau)A^T} y = 0
     \]
      bzw.
    \[
     0=y^T e^{(T_1-\tau)A} B =: K(\tau), \; \tau \in [T_0,T_1].
    \]
    Es ist 
    \begin{align*}
      K'(\tau)=y^T e^{(T_1-\tau)A} B \cdot (-A) \\
      K'(T_1)=y^T B (-A).
    \end{align*}
    i-faches Differenzen von $K(\tau)$ nach $\tau$ ergibt
    \begin{align*}
      K^{(i)}(\tau)=y^T e^{(T_1-\tau)A} \cdot B (-1)^i A^i \\
      K^{(i)}(T_1)=(-1)^i y^T A^i B
    \end{align*}
    Da $K \equiv 0$ auf $[T_0,T_1]$ gilt auch 
    \[
      K^{(i)}(T_1)=(-1)^i y^T A^i B = 0 \; \forall i \in \N.
    \]
    Insbesondere gilt:
    \[
     y^T A^i B = 0 \; \forall i \in \{0,\dots,n-1\}
    \]
    Das heißt für die Transponierte 
    \[
    	(A^i B)^T y = 0 \; \forall i \in \{0, \dots, n-1\}
    \]
    und damit auch
    \[
    	 (A^0 B | A^1 B | \dots | A^{n-1} B)^T \cdot y= 
    	 \begin{pmatrix}
    	 			(A^0 B)^T \\
    	 			(A^1 B)^T \\
    	 			\hdots \\
    	 			(A^{n-1} B)^T
    	 \end{pmatrix}
    	 \cdot y=0 \in \R^{n^2}
    \]
    Also hat $(A^0 B | A^1 B | \dots | A^{n-1} B)^T$ keinen trivialen Kern, somit keinen vollen Rang. Da der  Zeilenrang gleich dem Spaltenrang ist, muss auch $(A^0 B | A^1 B | \dots | A^{n-1} B)$ nicht vollen Rang haben. Das ist aber ein Widerspruch zur Kalman Rang Bedingung.
   \item["`$\Rightarrow$"']
    Die Rückrichtung wird wieder per Kontraktion gezeigt, die Beweisidee ist dieselbe wie in der Hinrichtung. 

    Angenommen, die Kalman Rang Bedigung wäre nicht erfüllt. Dann existiert ein $y \in \R^n$ mit $y^T A^i B = 0 \; \forall i \in \{0,\dots,n-1\}$.

    Durch Induktion lässt sich folgende Behauptung beweisen:
    Alle Matrizen $A^m$ lassen sich durch eine Linearkombination $\sum\limits_{i=0}^{n-1} \alpha^i A^i $ darstellen
    \begin{enumerate}
      \item[(IA)]
      Betrachte das charakteristische Polynom $\chi_A$ der Matrix $A$.
      \[
	\chi_A = X^n+\alpha_{n-1}X^n-1+\dots+\alpha_0
      \]
      Nach Cayley Hamilton gilt $\chi_A(A)=0$, also
      \[
      0=A^n+\sum\limits_{i=0}^{n-1} \alpha_i A^i \Leftrightarrow A^n = -\sum\limits_{i=0}^{n-1} \alpha_i A^i  
      \]

      \item[(IV)]
      Es gelte die Behauptung für ein beliebiges aber festes $m \in \N$ und alle $l<m$. 

      \item[(IS)]
      $m \mapsto m+1$ 
      \begin{align*}	
      A^{m+1} &= A^m \cdot A \overset{IV}{=} (\sum\limits_{i=0}^{n-1} \alpha_i A^i) A = \sum\limits_{i=0}^{n-1} \alpha_i A^{i+1} = \sum\limits_{i=1}^{n} \alpha_{i-1} A^i \\
							&= \sum\limits_{i=1}^{n-1} \alpha_{i-1} +\alpha_{n-1} A \overset{IA}{=} \sum\limits_{i=1}^{n-1} \alpha_{i-1} A + \sum\limits_{i=0}^{n-1} \beta_i A^i= \sum\limits_{i=0}^{n-1}\gamma_i A^i
		\end{align*}
		\end{enumerate}
		Es folgt die Behauptung per vollständiger Induktion.		
        
		Damit lässt sich für alle $m \in \N, m>n-1$ folgern:
		\[
			y^T A^m B = y^T (-\sum\limits_{i=0}^{n-1}\alpha_i A^i) B = -\sum\limits_{i=0}^{n-1} \alpha_i \underbrace{(y^T A^i B)}_{=0}=0
		\]
		Sei wieder 
		\[
			K(\tau):=y^{T e(T_1-\tau)A} B \; \tau \in [T_0,T_1].
		\]
		Dann ist 
		\[
			K^{(i)}(T_1)= (-1)^i y^T A^i B = 0
		\]
		$K$ ist analytisch, bildet man nun die Taylorreihe um $T_1$
		\[
			K(\tau) = \sum\limits_{i=0}^\infty \frac{K^{(i)}(T_1)}{i!} (\tau-T_1)^i = 0
		\]
		Es ergibt sich aus $\int_{T_0}^{T_1} |K(\tau)|^2 d\tau = 0$, dass
		\[ 
			\int\limits_{T_0}^{T_1} |B^T e^{(T_1-\tau)A^T} y | ^2 d\tau=0
		\]
		und damit auch
		\[
		y^T \G y = 0
		\]
		Da $\G$ semipositiv definit und symmetrisch ist, ist durch $(x,y) = x^T \G y$ eine symmetrische semipositive Bilinearform gegeben und auf diese die Cauchy-Schwarzsche Ungleichung anwendbar: Sei $z \in \R^n$
		\[
			0 \leq |y^T \G z| \leq \underbrace{\sqrt{y^T \G y}}_{=0} \cdot \sqrt{z^T \G z}=0
		\]
		Damit ist 
		\[
			z^T \G y = 0 \forall z \in \R^n,
		\]
		was bedeutet
		\[
			\G y = 0
		\]
		Da $y \neq 0$ ist $\G$ nicht invertierbar, also das System nicht kontrollierbar.
 \end{enumerate}
\end{Beweis}

Jetzt soll das Ergebnis auf den zeitabhängigen Fall übertragen werden. Dafür wird zunächst induktiv die Folge $\{ B_i\}_{i \in \N} \subset \mathcal{C}^\infty ([T_0,T_1])$ wie folgt definiert.
\[
	B_0(t)=B(t); \qquad B_i(t)=B_{i-1}(t) - A(t) \cdot B{i-1}(t) \; \forall t \in [T_0,T_1]
\]
Damit erhält man folgenden Satz

\begin{Satz}\label{Kalman Rang Bedingung zeitabhänig}
Sei ein $\overline t \in [T_0,T_1]$ und $\{i_1,\dots i_n\}\subset \N$ mit
\[
	\rang(B_{i_1} \mid B_{i_2} \mid \dots \mid B_{i_{n}})=n 
\]
Dann ist das System $\dot x = A(t) x +B(t) u$ kontrollierbar in $[T_0,T_1]$
\end{Satz}

\begin{Bemerkung}
Im Gegensatz zum zeitunabhängigen Fall ist jetzt nur noch eine Folgerung gegeben. Und statt der vorher zu prüfenden Matrizen $A^0B, \dots A^{n-1}B$ gibt es jetzt mehr Freiheiten für die Wahl der Matrizen BLBLABA
\end{Bemerkung}

\begin{Beweis}[zu \ref{Kalman Rang Bedingung zeitabhänig}]
Annahme: $\dot x = A(t) x + B(t)u$ ist nicht kontrollierbar. Dann ist auch $\G$ nicht invertierbar und hat damit keinen trivialen Kern. Es exisiert also $y \in \R^n, y \neq 0$ mit $\G y =0$.
Also
\[
	0 = y^T \G y = \int\limits_{T_0}^{T_1} \mid B(\tau)^T \cdot R(T_1,\tau)^T \cdot y \mid^2 d\tau 
\]
Wiederum folgt aus der Stetigkeit
\[
	B(\tau)^T\cdot R(T_1,\tau)^T \cdot y = 0 \; \forall \tau \in [T_0,T_1]
\]
Indem $z:=R(T_1,\overline t)^T y$ definiert wird, kommt das $\overline t $ zum Einsatz. Zu bemerken ist, dass aus (\ref{Resolvente Eigenschaften:3}) folgt, dass $R(T_1,\overline t)$ invertierbar ist und damit dann auch $R(T_1,\overline t)^T$. Da $y \neq 0$, ist auch $z \neq 0$.\\
Definiere jetzt
\[
	K(\tau):=z^T R(T_1,\tau) B(\tau) \overset{\ref{Resolvente Eigenschaften:2}}{=} y^T R(T_1,\overline t) R(\overline t, \tau) B(\tau) = 0 \; \forall \tau \in [T_0,T_1]
\]
Ableiten von $K(\tau)$ mithilfe von (\ref{Resolvente Ableitung:2}), und der Produktregel ergibt
\begin{align*}
	K'(\tau) =& z^T(-R(\overline t, \tau)A(\tau)B(\tau))+z^T R(\overline t, \tau)\dot B(\tau) \\
	=& z^T R(\overline t, \tau)(-A(\tau)B(\tau))+ \dot B(\tau)) \\
	=& z^T R(\overline t, \tau) B_1(\tau)	
\end{align*}

Wenn man nun $(i)$-mal differenziert
\begin{align*}
	\frac d {d\tau^i} K (\tau) =& z^T (A(\tau) R(\overline t, \tau) B_i(\tau)+R(\overline t, \tau) \dot B_i(\tau)\\
	=& z^T R(\overline t, \tau) (\dot B_i(\tau)-(A(\tau)B_i(\tau)\\
	=& z^T R(\overline t, \tau) B_{i+1}(\tau)
\end{align*}

\end{Beweis}

\section{Hilbert-Eindeutigkeits-Methode}

Nachdem wir also gesehen haben, wann ein lineares 
Kontrollsystem  kontrollierbar ist und wie dann eine, unter
einem gewissen Aspekt, optimale Steuerung aussieht, wollen wir uns
einen etwas allgemeineren Rahmen der Problemstellung widmen. Dazu
sei daran erinnert, wie in Beweis \ref{Kontrollierbarkeit} die Steuerung
$\bar{u}$ definiert wurde:
$$
\bar{u}(\tau):=B(\tau)^TR(T_{1},\tau)^T\mathfrak{C}^{-1}(x_{1}-R(T_{1},T_{0})x_{0}),~~\forall\tau\in[T_{0},T_{1}]
$$


Bei der Berechnung dieser Steuerung treten einige Probleme auf, die
wir im Folgenden betrachten wollen.

Sei also weiter das Kontrollsystem
$$
\dot{x}(t)=A(t)x(t)+B(t)u(t),~~~~x(T_{0})=0
$$
zu Grunde gelegt.

\subsection*{Die Erreichbarkeitsmenge}

Die Steuerung $\bar{u}$ konnte nur definiert werden, da die Gramsche
Kontrollmatrix $\mathfrak{C}$ invertierbar war, was genau dann der
Fall ist, wenn das Kontrollsystem  kontrollierbar ist.
Dies ist nat"urlich im Allgemeinen nicht immer der Fall, wie das folgende
einfache Beispiel zeigt:
\begin{Beispiel}
[nicht kontrollierbares System]\label{bsp1}

Sei ein zeitunabh"angiges lineares Kontrollsystem gegeben durch die
Matrizen
$$
A=\begin{pmatrix}1 & 0\\
0 & 1
\end{pmatrix}\text{\,\,\,\ und\,\,}\, B=\begin{pmatrix}1\\
1
\end{pmatrix}
$$
Dann folgt direkt mit Hilfe des Kalman-Rang-Kriteriums, dass das System
nicht kontrollierbar ist, da $A^{i}=A,\,\forall i\in\mathbb{N}$
und damit $rang(B\,|\, AB)=1$.
\end{Beispiel}
Dar"uber hinaus ist selbst bei kontrollierbaren Systemen
die Gramsche Kontrollmatrix nicht ohne weiteres berechenbar und/oder
der Aufwand f"ur Bestimmung und Invertierung in keinem akzeptablem
Rahmen (Rechenaufwand Matrixinvertierung $\mathcal{O}(n^{3})$). 

Um dennoch weitere Aussagen "uber solche Systeme treffen zu k"onnen,
die eben nicht kontrollierbar sind, f"uhren wir einen neuen
Begriff ein:


\begin{Definition}[Erreichbarkeitsmenge]
Wir definieren die Menge 
\begin{align*}
\mathcal{R}(t):=&\{x_{1}\in\mathbb{R}^{n}|\text{ es ex. eine Steuerung }u\\
&x\text{ L"osung des Kontrollsystems und es gilt }x(t)=x_{1}\}
\end{align*}
\end{Definition}

Das hei"st $\text{\ensuremath{\mathcal{R}}(t) ist die Menge der Zust"ande (oder Punkte), die von \ensuremath{x_{0}=0}}$
in der Zeit $t=T_{1}-T_{0}$ durch alle m"oglichen Steuerungen $u$
erreicht werden k"onnen.

F"ur unser Beispiel (\ref{bsp1}) l"asst sich die Erreichbarkeitsmenge
wie folgt bestimmen:

Die L"osung des Anfangswertproblems hat die Form
$$
x(t)=e^{A(t-T_{0})}x_{0}+\int_{T_{0}}^{t}e^{A(t-s)}Bu(s)ds=\int_{T_{0}}^{t}e^{A(t-s)}\begin{pmatrix}1\\
1
\end{pmatrix}u(s)ds
$$
Und damit, wegen der Diagonalit"at von $A(t)$, ergibt sich f"ur die
L"osungskomponenten
$$
\begin{pmatrix}x_{1}(t)\\
x_{2}(t)
\end{pmatrix}=\begin{pmatrix}\int_{T_{0}}^{t}e^{(t-s)}u(s)ds\\
\int_{T_{0}}^{t}e^{(t-s)}u(s)ds
\end{pmatrix}
$$
Das hei"st, beide L"osungskomponente sind f"ur jede Steuerung $u$ zu
jeder Zeit $t>T_{0}$ identisch:

%\vspace{5cm}
\begin{center}
\begin{figure}[ht]
\includegraphics[clip,scale=0.7]{R}\caption{Erreichbarkeitsmenge aus Beispiel (\ref{bsp1})}
\end{figure}
\end{center}
%\vspace{5cm}

Die Erreichbarkeitsmenge eines Kontrollsystems hat einige n"utzliche
Eigenschaften, die hier zwar kurz erw"ahnt werden sollen, jedoch nicht
bewiesen werden, da wir sie in unserem Kontext nicht verwenden.

\begin{Lemma}
Eigenschaften der Erreichbarkeitsmenge

\begin{itemize}
\item $\mathcal{R}(t)$ ist f"ur alle $t\geq0$ ein Untervektorraum des $\mathbb{R}^{n}$.
\item $\mathcal{R}(t)=\mathcal{R}(s)$ f"ur alle $s,t>0.$
\end{itemize}
\end{Lemma}
Das hei"st also, die Erreichbarkeitsmenge $x_{0}+\mathcal{R}(t)$ ist
ein affiner Unterraum mit der Dimension von $\mathcal{R}(t)$ und
$\mathcal{R}(t)$ ist zeitunabh"angig - man kann einen Zustand, falls
er erreichbar ist, in jeder Zeit erreichen. Wir schreiben im weiteren
Verlauf kurz $\mathcal{R}$ statt $\mathcal{R}(t)$.

\subsection*{Hilbert-Eindeutigkeits-Methode (HUM)}

Ziel ist es jetzt einen Weg zu finden, um eine Steuerungsfunktion
zu bestimmen, die unabh"angig von der Kontrollierbarkeit, leicht zu
berechnen und in einem noch zu definierenden Sinne \emph{optimal}
ist. Einen solchen Weg bzw. Algorithmus liefert uns die \emph{Hilbert-Eindeutigkeits-Methode(HUM)}.

Dazu f"uhren wir zun"achst ein Differentialgleichungssystem ein:

\begin{Definition}[adjungiertes System]


Wir definieren f"ur ein $ \phi_{1}\in\mathbb{R}^{n}$ das adjungierte
zeitumgekehrte homogene Kontrollsystem
$$
\dot{\phi}=-A(t)^T\phi,~~~~\phi(T_{1})=\phi_{1}
$$
und bezeichnen die L"osung mit $\phi(t)$.\end{Definition}

Die Bezeichnung \emph{zeitumgekehrt} wird durch die Tatsache gerechtfertigt, dass
die L"osung dieses Anfangswertproblems die L"osungsbahn des Anfangswertproblems  $\dot{\phi}=A(t)\phi,~~\phi(T_{1})=\phi_{1}$ genau umgekehrt durchl"auft.
Um nun die HUM formulieren zu k"onnen, ben"otigen wir die folgende, in
einem Korollar zusammengefasste, Identit"at:

\begin{Korollar}
Sei $x : [t_0,t_1] \rightarrow \mathbb R^n$ die L"osung des Anfangswertproblems
\begin{align*} \label{hum1} \tag{$\star$}
\dot x = A(t)x+B(t)u(t) ~~~~ \forall x(t_0)=0
\end{align*}
Sei au"serdem $\phi_1 \in \mathbb R^n$ und $\phi : [t_0,t_1] \rightarrow \mathbb R^n $ die L"osung des  adjungierten Systems
\begin{align*}
\dot \phi = -A(t)^T\phi ~~~~ \phi(t_1)=\phi_1
\end{align*}
Dann gilt:
\begin{align*}
x(t_1) \cdot \phi_1 = \int_{t_0}^{t_1} u(\tau) \cdot B(\tau)^T \phi(\tau) d\tau.
\end{align*}
\end{Korollar}

\begin{Beweis}
Vorbemerkung: Um die Aussage zu beweisen, wollen wir uns zun"achst klarmachen, dass mit dem Hauptsatz der Differential- und Integralrechnung gilt:
\begin{align*}
\int_{t_0}^{t_1} \frac{d}{d\tau} (x(\tau) \cdot \phi(\tau)) d\tau 
&= x(t_1) \cdot  \underbrace{\phi(t_1)}_{=\phi_1} - \underbrace{x(t_0)}_{=0} \cdot \phi(t_0) \\
&= x(t_1) \cdot \phi_1
\end{align*}
Dann ergibt sich Folgendes:
\begin{align*}
x(t_1) \cdot \phi_1 &= \int_{t_0}^{t_1} \frac{d}{d\tau} (x(\tau) \cdot \phi(\tau)) d\tau\\
&=\int_{t_0}^{t_1} \left( \frac{d}{d\tau} x(\tau) \cdot \phi(\tau) + x(\tau) \cdot \frac{d}{d\tau} \phi(\tau)\right) d\tau \\
&=\int_{t_0}^{t_1}( \underbrace{(A(\tau)x(\tau) + B(\tau)u(\tau))}_{= \dot x(\tau)} \cdot \phi(\tau) - x(\tau) \cdot \underbrace{A(\tau)^T \phi(\tau)}_{=\dot \phi(\tau)} ) d\tau\\
&=\int_{t_0}^{t_1}\left( A(\tau)x(\tau) \cdot \phi(\tau)\right)+\left(B(\tau)u(\tau) \cdot \phi(\tau) \right) - \left( x(\tau) \cdot A(\tau)^T \phi(\tau) \right) d\tau\\
&= \int_{t_0}^{t_1}\left( [A(\tau)x(\tau) ]^T \phi(\tau)\right)+\left([B(\tau)u(\tau)]^T \phi(\tau) \right) - \left( x(\tau)^T A(\tau)^T \phi(\tau) \right) d\tau\\
&=\int_{t_0}^{t_1}\left( x(\tau)^TA(\tau)^T \phi(\tau)\right)+\left(u(\tau)^TB(\tau)^T \phi(\tau) \right) - \left( x(\tau)^T A(\tau)^T \phi(\tau) \right) d\tau\\
&=\int_{t_0}^{t_1}\left(u(\tau)^TB(\tau)^T \phi(\tau) \right)d\tau \\
&=\int_{t_0}^{t_1}\left(u(\tau) \cdot B(\tau)^T \phi(\tau) \right)d\tau
\end{align*}
Dies war unsere Behauptung.
(Bemerkung: Wobei $a \cdot b := a^Tb$, also das Standardskalarprodukt ist) 
\end{Beweis}

Wir wollen jetzt die HUM als Operatorfunktion $\Lambda$ formulieren:
\begin{Definition}
Wir definieren folgende Abbildung:
\begin{align*}
\Lambda : \mathbb R^n \rightarrow \mathbb R^n, \phi_1 \mapsto x(t_1)
\end{align*}
Also bildet $\Lambda$ einen Startzustand von $\phi$ auf einen Endzustand von $x$ ab.
Wobei $x: [t_0,t_1] \rightarrow \mathbb R^n$ die L"osung des Anfangswertproblems
\begin{align*}
\dot x = A(t)x + B(t) \bar u(t) ~~~~ x(t_0)=0
\end{align*}
mit
\begin{align*}
\bar u(t):=B(t)^T \phi(t)
\label{baru} \tag{$\star\star$}
\end{align*}
und $\phi :[t_0,t_1] \rightarrow \mathbb R^n$ L"osung des adjungierten Systems
\begin{align*}
\dot \phi = -A(t)^T \phi ~~~~ \phi(t_1) = \phi_1 , ~ t\in [t_0,t_1]
\end{align*}
 f"ur ein beliebiges $\phi_1$ ist.
\end{Definition}

\begin{Korollar}
$\Lambda$ liefert uns also einen Algorithmus, um eine Steuerung $\bar u$ zu generieren, der wie folgt zu verstehen ist:
\begin{enumerate}
\item W"ahle ein $\phi_1 \in \mathbb R^n$ 
\item L"ose das Anfangswertproblem $\dot \phi = -A(t)^T \phi, ~ \phi(t_1) = \phi_1$
\item Berechne $\bar u(t)=B(t)^T \phi(t)$, wobei $\phi(t)$ die L"osung des in Schritt 2 gel"osten Anfangswertproblems ist.
\item L"ose das Anfangswertproblem $\dot x = A(t)x + B(t) \bar u(t), ~x(t_0)=0$
\end{enumerate}
\end{Korollar}

\begin{Bemerkung}[Eindeutigkeit]
Die Abbildung $\Lambda$ ist injektiv, da sowohl in Schritt 2, also auch in Schritt 4 eindeutig l"osbare Anfangswertprobleme berechnet werden, somit ist $\bar u$ eindeutig bestimmt und damit wiederum wird jedem $\phi_1$ genau ein $x(T_1)$ zugeordnet.
\end{Bemerkung}

Was die Bemerkung bereits andeutet und den Vorteil der HUM ausmacht, fasst das folgende Theorem zusammen:
\begin{Satz}[Teil 1]
Mit den Vorausetzungen der vorangegangen Definition gilt:
\begin{align*}
\mathcal R = \Lambda(\mathbb R^n)
\end{align*}
Das hei"st also, dass das Bild der Abbildung $\Lambda$ genau gleich der Erreichbarkeitsmenge $\mathcal R$ des Kontrollsystems (\ref{hum1}) ist.


\end{Satz}


\begin{Beweis}

"$\supseteq$":
Dies folgt direkt aus der Definition von $\Lambda$, da $\bar u \in L^{\infty}$ 
\\
"$\subseteq$":
Zu zeigen ist, dass ein beliebiges $x_1 \in \mathcal R$ im Bild der Abbildung $\Lambda$ liegt.
Sei also $x_1 \in \mathcal R$ beliebig, das hei"st es existiert eine Sterung $u^*$, so dass die L"osung $x^*(t)$ sowohl das Anfangswertproblem $$\dot x = A(t)x + B(t)  u^*(t), ~x(t_0)=0$$
als auch $$x^*(t_1)=x_1$$ erf"ullt.

Sei jetzt $\mathcal U \subset C^0([T_0,T_1], \mathbb R^m)$ definiert durch: $$ \mathcal U := \{ u \in C^0([T_0,T_1], \mathbb R^m) | ~ u(t) = B(t)^T\phi(t) \} $$
mit $\phi(t)$ L"osung des adjungierten Systems $\dot \phi = -A(t)^T \phi, ~ \phi(t_1) = \bar\phi $ f"ur ein beliebiges $\bar\phi \in \mathbb R^n$.
Um sp"ater eine orthogonale Projektion auf $\mathcal U$ definieren zu k"onnen, m"ussen wir zeigen, dass $\mathcal U$ ein endlich-dimensionaler Untervektorraum (UVR) ist:
\begin{Lemma}
$\mathcal U$ ist ein Untervektorraum.
\end{Lemma}
\begin{Beweis}
\begin{enumerate}
\item[ ]
\item $0 \in \mathcal U$, da f"ur $\phi_1=0$ gilt: $\phi(t) \equiv 0$, damit ist $\bar u(t) =B(t) \phi(t) \equiv 0$
\item Seien $u_1,u_2 \in \mathcal U$, dann ist zu zeigen, dass $u_1+u_2$ ebenfalls in $\mathcal U$ liegt:
\begin{align*}
u_1+u_2 = B(t)^T \phi_1(t) + B(t)^T\phi_2(t) = B(t)^T  \underbrace{(\phi_1(t) +\phi_2(t))}_{=\phi(t)\text{, da Lsg. VR}} \in \mathcal U
\end{align*}
\item F"ur ein $u\in \mathcal U$ und ein $\lambda \in \mathbb R$ gilt $\lambda u \in \mathcal U$.
\end{enumerate}
Damit ist $\mathcal U$ ein Untervektorraum von $C^0([T_0,T_1], \mathbb R^m)$ mit $dim(\mathcal U) \leq n$.
\end{Beweis}
Nachdem wir den Hilfssatz bewiesen haben, jetzt weiter zum unspr"unglichen Beweis unseres Theorems:
Sei also $\bar u$ die orthogonale Projektion von $u^*$ auf $\mathcal U$.

\begin{Exkurs*}

Orthogonale Projektion: Sei $V$ ein endlich dimensionaler Vektorraum mit positiv definitem Skalarprodukt und $U$ ein Untervektorraum von $V$, dann ist die orthogonale Projektion auf $U$ eine Abbildung $p : V \rightarrow V$ mit:

F"ur alle $v \in V$ gilt
\begin{enumerate}
\item $p(v) \in U$
\item $v-p(v) \in U^{\perp}$
\end{enumerate}
\end{Exkurs*}
  
Dann gilt:
\begin{align*}
\int \underbrace{(u^*-\bar u) \cdot u}_{=0\text{, da ortho.}} = 0, ~~~~ \forall u \in \mathcal U
\end{align*}
$$\Leftrightarrow$$
\begin{align*}
\int u^* \cdot u = \int \bar u \cdot u,~~~~ \forall u \in \mathcal U
\end{align*}

Sei weiter $\bar x : [t_0,t_1]\leftarrow \mathbb R^n$ L"osung des Anfangswertproblems 
\[
\dot x = A(t)x + B(t) \underbrace{\cdot \bar u(t),}_{\text{orth. Proj.}} ~~~ x(t_0)=0
\]
 Dann gilt:
\begin{align*}
x_1 \cdot \phi_1 = x^*(t_1) \cdot \phi_1 &= \int_{t_0}^{t_1} u^* \cdot B(t)^T \phi(t) dt \\
&=\int_{t_0}^{t_1} \bar u \cdot B(t)^T \phi(t) dt \\
&=\bar x(t_1) \cdot \phi_1
\end{align*}
$\Rightarrow x_1 \cdot \phi_1 = \bar x(t_1) \cdot \phi_1 \Rightarrow \bar x(t_1) = x_1$

Da $\bar u \in \mathcal U$ existiert ein $\bar \phi_1 \in \mathbb R^n$, so dass $\bar \phi (t)$ L"osung des Anfangswertproblems  $\dot \phi = -A(t)^T \phi, ~ \phi(t_1) = \bar\phi_1 $. $\Lambda$ ist injektiv, damit folgt direkt  $x_1 = \Lambda(\bar\phi_1) \in \Lambda(\mathbb R^n)$ und somit ist die Inklusion gezeigt.

Und damit ist der erste Teil des Theorems gezeigt. Es gilt also $$\mathcal R = \Lambda(\mathbb R^n)$$
\end{Beweis}
Das hei"st wir haben eine M"oglichkeit gefunden einem jeden Punkt des $\mathbb R^n$ genau einen Punkt der Erreichbarkeitsmenge zuzuordnen. Zudem l"asst sich die Steuerung $\bar u$ leichter berechnen und ist unabh"angig von der Gramschen Kontrollmatix. Jetzt wollen wir noch zeigen, dass unser $\bar u$ die selbe Steuerung wie in Beweis zu \ref{Kontrollierbarkeit} ist, jedoch in einem allgemeinerem Rahmen, n"amlich ohne Regularit"at von $\mathfrak C$:

Bereits in Beweis zu \ref{Kontrollierbarkeit} hatten wir gesehen, dass unsere Steuerung in gewissen Sinne optimal ist und dies sogar als einzige Steuerung. Also m"ussen wir nur zeigen, dass unsere, durch die HUM generierte Steuerung ebenfalls optimal  ist:
\begin{Satz}[Teil 2]
Wenn $x_1= \Lambda(\phi_1)$ und $u$ eine weitere Kontrollfunktion ist, welche das Kontrollsystem $\dot x = A(t)x+B(t)u,~ t\in[t_0,t_1]$ von $x_0=0$ nach $x_1 \in \mathbb R^n$ w"ahrend des Zeitintervalls $[t_0,t_1]$ steuert. Dann gilt:
\begin{align*}
\int_{t_0}^{t_1} \|\bar u(\tau) \|^2 d\tau \leq \int_{t_0}^{t_1} \| u(\tau) \|^2 d\tau
\end{align*}
Wobei $\bar u$ wie in (\ref{baru}) und Gleichheit genau dann gilt, wenn $ u \equiv \bar u$.
\end{Satz}

\begin{Beweis}
Sei $v:=u-\bar u$ und damit $u=\bar u + v$, dann gilt wegen der Orthogonalit"at:
\begin{align*}
\int_{t_0}^{t_1} \| u(\tau) \|^2 d\tau = \int_{t_0}^{t_1} \|\bar u(\tau) \|^2 d\tau + \underbrace{\int_{t_0}^{t_1} \|v(\tau) \|^2 d\tau}_{\geq 0} + 2\cdot \underbrace{\int_{t_0}^{t_1} \| \bar u(\tau)\cdot v\|^2 d\tau}_{=0 \text{, orth. Proj.}}
\end{align*}
Insbesondere gilt $v\equiv 0$ nur, genau dann wenn $u \equiv \bar u$ ist.

Unser $\bar u$ ist also optimal und erf"ullt damit die Gleichheit zur Steuerung aus Beweis zu \ref{Kontrollierbarkeit}.
\end{Beweis}

\newpage
\printbibliography
\newpage
\printindex
\end{document}

% ===REFERENZEN===
% Krieg - Gew. DGl
% Rannacher
