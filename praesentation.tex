\documentclass{beamer}

\usetheme{Warsaw}             % Falls Ihnen das Layout nicht gef�llt, k�nnen Sie hier
                              % auch andere Themes w�hlen. Ein Verzeichnis der m�glichen 
                              % Themes finden Sie im Kapitel 15 des beameruserguide.

\usepackage[utf8]{inputenc} % um Umlaute direkt eingeben zu

\usepackage[utf8]{inputenc}
\usepackage[ngerman]{babel}
\usepackage{amsmath}
\usepackage{amsfonts}
\usepackage{amssymb}
\usepackage{graphicx}

\AtBeginSection[]{\frame<beamer>{\frametitle{Inhalt} \tableofcontents[current]}}

\newcommand{\defin}[1]{\textit{\color{blue}#1}}

% ========== Abk�rzungen ==========
\newcommand{\N}{\mathbb{N}}
\newcommand{\Z}{\mathbb{Z}}
\newcommand{\Q}{\mathbb{Q}}
\newcommand{\R}{\mathbb{R}}
\newcommand{\C}{\mathbb{C}}
\newcommand{\rang}[0]{\operatorname{rang}}

\author{Niklas Fischer, Elisa Friebel, Marcel Goesmann}
\title{Einführung in die Kontrolltheorie}
\date{\today \\[.5\baselineskip] Vortrag zum Seminar Modellierung und Simulation}

\begin{document}
\frame{\maketitle}

\section{Kalman Rang Bedingung}
  \begin{frame}
  Resolventen sind im Allgemeinen schwer zu bestimmen. 
  \begin{block}{Frage}
  Gibt es eine einfachere Möglichkeit Systeme auf Kontrollierbarkeit zu überprüfen?
  \end{block}
\end{frame}

\subsection{zeitunabhängiger Fall}

\frame{\frametitle{}
\begin{block}{Kalman Rang Bedingung}
 Das lineare, zeitunabhängige System $\dot{x}(t)=Ax(t)+Bu(t)$ ist in $[T_0,T_1]$ kontrollierbar genau dann wenn
\[
  \rang(A^0B \mid A^1B \mid A^2B \mid \dots  \mid A^{n-1}B)=n
\]
\end{block}
\pause
\begin{itemize}
 \item $A \in \R^{n \times n}, \; B \in \R^{n \times m} \pause \Rightarrow (A^0B \mid \dots  \mid A^{n-1}B) \in \R^{n \times n\cdot m}$
 \pause
 \item Kontrolle des Systems ist unabhängig vom Intervall $[T_0,T_1]$
\end{itemize}

}

\subsection{zeitabhängiger Fall}

\frame{\frametitle{}

Sei die Folge $\{B_n\}_{n \in \N}\subseteq \R^{n \times m}([T_0,T_1])$ wie folgt induktiv definiert:
\[
	B_0(t)=B(t); \qquad B_i(t)=\dot B_{i-1}(t) - A(t) \cdot B_{i-1}(t) \; \forall t \in [T_0,T_1]
\]
\pause

\begin{block}{Kalman Rang Bedingung im zeitabhängigen Fall}
 Wenn ein $\overline t \in [T_0,T_1]$ und $\{i_1,\dots i_k\}\subset \N$ für ein $k\in \N$ existieren, so dass
\[
	\rang(B_{i_1} (\overline t)\mid B_{i_2}(\overline t) \mid \dots \mid B_{i_{k}}(\overline t))=n, 
\]
dann ist das System
\[
  \dot x(t) = A(t) x(t) +B(t) u(t)
\]
kontrollierbar in $[T_0,T_1]$.
\end{block}
}


\begin{frame}
\frametitle{Vergleich des zeitabhängigen und zeitunabhängigen Falls}
	\begin{columns}[t]
		\column{5cm}
		   \begin{block}{zeitunabhängiger Fall}
         Das lineare, zeitunabhängige System $\dot{x}(t)=Ax(t)+Bu(t)$ ist in $[T_0,T_1]$ kontrollierbar \textbf{genau dann wenn}
        \[
          \rang(A^0B \mid \dots  \mid A^{n-1}B)=n
        \]
      \end{block}
		
		\column{5cm}
		   \begin{block}{zeitabhängiger Fall}
          Wenn ein $\overline t \in [T_0,T_1]$ und $\{i_1,\dots i_k\}\subset \N$ für ein $k\in \N$ existieren, so dass
          \[
	          \rang(B_{i_1} (\overline t) \mid \dots \mid B_{i_{k}}(\overline t))=n, 
          \]
dann ist das System $\dot x(t) = A(t) x(t) +B(t) u(t)$ kontrollierbar in $[T_0,T_1]$
        \end{block}

	\end{columns}
	\pause
	\begin{itemize}
	\item im zeitabhängigen Fall ist nur noch eine Folgerung gegeben
	\pause
	\item es gibt unedlich viele Möglichkeiten für $\overline t$ und $\{i_1,\dots i_k\}$
	\end{itemize}
\end{frame}


%}

%
%\begin{exampleblock}{(1.4) Beispiel}
%Die Funktion $f: \R \to \R$, $x \mapsto 2 \cdot x + 3$ ist stetig, da die Identit�t und die konstanten Funktionen stetig sind.
%\end{exampleblock}
%}
%
%\frame{\frametitle{Definition: gleichm��ige Stetigkeit}
%\begin{block}{(1.6) Definition}
%Eine Funktion $f: D \to \R$ mit $D \subset \R$ hei�t \defin{gleichm��ig stetig}, wenn zu jedem $\varepsilon > 0$ ein $\delta > 0$ %existiert, so dass f�r alle $x, y \in D$ mit $|x - y| < \delta$ stets $|f(x) - f(y)| < \varepsilon$ gilt.
%\end{block}
%
%\vspace*{3ex}
%\uncover<2->{Es macht keinen Sinn, von einer gleichm��igen Stetigkeit in einem Punkt zu sprechen.}
%}

%\section{Zweiter Abschnitt}

\end{document}
