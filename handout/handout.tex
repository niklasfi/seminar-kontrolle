\documentclass[a4paper,twoside]{article}

\usepackage[utf8]{inputenc}
\usepackage{framed}
\usepackage[framed,thmmarks]{ntheorem}
\usepackage{amsmath}
\usepackage{amssymb}
\usepackage[T1]{fontenc}
\usepackage{multicol}
%\usepackage{cmbright}
\usepackage[a4paper,left=31mm,right=31mm,top=25mm,bottom=29mm]{geometry}
%\usepackage{fullpage}
 
\theoremstyle{nonumberplain}
\theorembodyfont{}
\newframedtheorem{Definition}{Definition}
\newframedtheorem{Lemma}{Lemma}

\newcommand{\R}[0]{\mathbb{R}}
\newcommand{\G}[0]{\mathfrak{C}}
\newcommand{\E}[1][n]{\mathrm{E_{#1}}}
\newcommand{\Id}[1][n]{\mathrm{Id_{#1}}}


\begin{document}
\begin{Definition}[Lineares Kontrollsystem]\index{lineares Kontrollsystem}
		Ein System
		\begin{align*}%\label{gl:ks}
			\dot x(t) = A(t) x(t) + B(t) u(t),\quad t \in [T_0, T_1]
		\end{align*}
		wird \emph{lineares Kontrollsystem} genannt.
	\end{Definition}
	
	\begin{Definition}[kontrollierbar]\index{kontrollierbar}
		Ein lineares Kontrollsystem 
			\[
				\dot x(t) = A(t)x(t) + B(t)u(t)
			\]
		ist \emph{kontrollierbar}, falls es für alle $x_0,x_1 \in \R^n$ eine stetige Funktion $u$ gibt, sodass die Lösungsfunktion $x$ des Anfangswertproblems 
		\begin{align*}
			\dot x(t) = A(t)x(t) + B(t)u(t) \; \forall t \in [T_0, T_1] \quad \text{mit} \quad x(T_0) = x_0
		\end{align*}
		in $T_1$ den Wert $x_1$ annimmt.
	\end{Definition}
	
		\begin{Definition}[Resolvente]
			Die \emph{Resolvente} eines linearen Systems ist definiert als
			\[
				R: [T_0, T_1] \times [T_0, T_1] \rightarrow \R^{n \times n},
				(t, \tau) \mapsto \tilde R(\tau)(t),
			\]
			 wobei $\tilde R(\tau)$ das homogene Anfangswertproblem 
			\[
			\dot M(t) = A(t)\cdot M(t) \; \forall t \in [T_0, T_1]  \quad \text{mit} \quad M(\tau) = \E
			\] löst.
		\end{Definition}


\begin{Lemma}[Eigenschaften der Resolvente]
\mbox{}
\columnsep0pt
\begin{multicols*}{2}
\begin{enumerate}
	\item $R$ ist stetig
	\item $	R(t_1,t_1) = \E$
	\item $ R(t_1, t_2) \cdot R(t_2, t_3) = R(t_1, t_3)$
	\item $ R(t_1, t_2) \cdot R(t_2, t_1) = \E$ 
	\item $ \frac{\partial R}{\partial t} (t, \tau) = A(t) \cdot R(t,\tau) \; \forall t,\tau \in [T_0, T_1]$
	\item $ \frac{\partial R}{\partial \tau} (t, \tau) = - R(t, \tau) \cdot A(\tau) \; \forall t,\tau \in [T_0, T_1]$
	\item $x(t_1) = R(t_1, t_0) x(t_0) + \int_{t_0}^{t_1} R(t_1,\tau) b(\tau) d\tau$
	\item $x(t) = R(t, T_0) x^0+ \int_{T_0}^t R(t, \tau)  b(\tau)d\tau$.
\end{enumerate}
\end{multicols*}

\end{Lemma}

\begin{Definition}[Gram'sche Kontrollmatrix]\index{Gram'sche Kontrollmatrix}
			Für ein Kontrollsystem \[
				\dot x(t) = A(t) x(t) + B(t) u(t)
			\]
			heißt die Matrix \[
				\G:=\int_{T_0}^{T_1} R(T_1, \tau) \cdot B(\tau) \cdot B(\tau)^T\cdot R(T_1, \tau)^T d\tau \in \R^{n \times n}
			\] Gram'sche Kontrollmatrix.
		\end{Definition}
		
\end{document}
